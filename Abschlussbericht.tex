\documentclass{article}


\begin{document}
	
	\section{Einleitung}
	
		\subsection{Motivation und Problemstellung}
		
		\subsection{Verwandte Arbeiten}
		Falls zu groß/zu viel, auslagern in ein eigenes Kapitel nach Grundlagen. Beschreiben warum andere Quellen nicht ausreichend waren und weshalb der eigene Ansatz jene fehlende Themen ergänzt.
		
		\subsection{Zielsetzung und Vorgehensweise}
		
		\subsection{Übersicht}
	
	
	\section{Grundlagen}
	(Die benutzten) Vortragsthemen vom Anfang hier als eigene Unterkapitel beschreiben.
		
		
	\section{Fachliches Vorgehen}
	Keine technischen Details (wie z.B. Implementierung)
	
		\subsection{Projektorganisation}
		Wie sind wir vorgegangen... auf alles bezogen. Wer hat an welchen Kapiteln mitgearbeitet.
	
		\subsection{Creature Generation}
		
		\subsection{Creature Animation}
		
		\subsection{Terraingeneration}
		
	\section{Technische Umsetzung}
	Eingehen auf Status Quo.
	
	\section{Vorläufige Ergebnisse}
	Unterkapitel nach Erkenntnissen. Metrik nach der bewertet wird erörtern. Objektiv ohne Wertung der Ergebnisse.
	
		\subsection{Diskussion}
		Diskussion der Ergebnisse in Bezug auf die initiale Zielsetzung.
	
	\section{Ausblick}
	
	
\end{document}
\chapter*{Zusammenfassung}

Bei der Entwicklung eines Videospiels kommen prozedurale Inhaltsgenerierung und Deep Reinforcement Learning immer häufiger zum Einsatz. Nicht nur können damit Zeit und Ressourcen gespart werden, sondern auch Flexibilität in dem Entwurf des Spieles aufrechterhalten werden. Die Wahl geeigneter Verfahren für die Kreaturen- und Inhaltsgenerierung und für das Animieren der Kreaturen bildet dabei oft eine zentrale Herausforderung. Im Rahmen einer Projektgruppe der TU-Dortmund, nehmen sich somit 11 Teilnehmer über einen Zeitraum von 2 Semestern der Aufgabe an, in insgesamt 3 Untergruppen, den Creature-Generatorn, World-Generatorn und Creature-Animatorn, einen Prototypen eines Videospiels mittels prozeduraler Inhaltsgenerieung und Deep Reinforcement Learning zu entwickeln. In dieser Ausarbeitung werden für die Kreaturen- und Inhaltsgenerierung, Methoden wie das L-System, Space-Partitioning, Metaball-Generierung, die Bone-Heat-Methode für automatisches Rigging und weiterhin eigen entwickelte, parametrische Methoden näher betrachtet. Währenddessen werden für das Animieren der Kreaturen ein RL-Framework namens NeroRL, Ansätze des ML-Agents von Unity und LiDO3 genutzt. Dabei werden während des Zeitraumes der Entwicklung sowohl technische Umsetzungen und die weitere Erforschung der jeweiligen Fachgebiete erörtert, als auch die organisatorischen Mittel und Entscheidungen der Mitglieder über den Verlauf der Implementierung des Spiels festgehalten. 

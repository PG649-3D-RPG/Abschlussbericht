\section{Motivation und Problemstellung}

Das Modellieren einer Kreatur in der Videospielindustrie wird üblicherweise von Spieldesignern vom Skelett, über die Haut und die Texturen welche auf der Haut sichtbar sind vormodelliert. Dieser Prozess ist für eine Vielzahl von sich variierenden Spielcharakteren jedoch sehr mühsam und kostet viel Zeit. Daher spielt die prozedurale Generierung von NPCs, sowie das Erlernen von Bewegungen und das (automatisierte) Animieren dieser eine immer größer werdende Rolle und stellt damit auch eine zentrale Herausforderung dar. Kreaturen müssen effizient und somit oftmals zur Laufzeit generiert werden und weiterhin die durch maschinelles Lernen angeeigneten Animationen auf ein möglichst breites Spektrum an Kreaturen anwendbar sein. Somit ist es relevant, dass das Training der Kreaturen möglichst generalisiert stattfindet, sodass bei der Kreaturen-Generierung eine große, sich bei den Körpermerkmalen variierende Menge der Kreaturen erzeugt werden kann, sodass durch die Animation möglichst wenig Einschränkungen für die Körpereigenschaften aufkommen. Somit sollten die Animationen auf einen möglichst breiten Pool von Kreaturen anwendbar sein, sodass für Kreaturen mit neuen Körpermerkmalen nicht neu trainiert werden muss.

Eine zentrale Herausforderung bei der Generierung von Spielfiguren ist das automatische Aufspannen eines Meshes über den bis zu diesem Zeitpunkt untexturierten Körper der Kreatur; das sogenannte \textit{Automatic Rigging}, welches ebenfalls zu dem prozeduralen Erzeugen von Kreaturen dazugehört. Es muss hier das Problem betrachtet werden, Meshes welche ebenfalls prozedural erzeugt werden, auf das breite Spektrum von verschiedenen Körperausprägungen des Kreaturen-Pools anwenden zu können, ohne dass es zu sehr von den Körperformen der Kreaturen abweicht.

Ferner kann die prozedurale Generierung von Inhalten in einem Videospiel nicht nur zum Erzeugen von Kreaturen zum Einsatz kommen, sondern auch für die Spielwelt selbst. 

In dieser Arbeit wird für die Generierung von Skeletten für Kreaturen, das Rigging, eine parametrisierbare Methode implementiert, die sich an dem Paper \cite{Hudson2013CreatureGU} von Jonathan A. Hudson orientiert. Das Erzeugen eines Meshes für die Skelette wird mit einer Metaball-Generierung (TODO: hauseigenen oder Referenz?) erreicht. Für das anschließende Zuordnen des Meshes zu den Komponenten des Skeletts, das Skinning, wird sowohl das Dual-Quaternion-Skinning (cite TODO), als auch die Bone-Heat-Method \cite{bone_heat_paper} betrachtet. 

Für das Animieren, das automatische Fortbewegen und Verhalten der Kreaturen, wird maschinelles Lernen verwendet; das Sammeln von Observationen des Verhaltens der Kreaturen wird innerhalb von Unity in der Movement Arena mittels einer ML-Agents Agent \cite{mlAgents} Schnittstelle umgesetzt. Das Verarbeiten und Training aus den resultierenden Observationen wird entweder durch ein Netz aus einer ONNX Datei oder durch NeroRL \cite{neroRL}, ein Python Backend, realisiert.

Für die Generierung der Spiellevel und dessen restlichen Inhalte der gesamten Spielwelt wird sowohl ein Space-Partitioning Algorithmus (\ref{Shaker2016}), als auch ein Algorithmus basierend auf dem Lindenmayer-System (L-System) \cite{lindenmayer1990} entwickelt. Ergänzend werden hauseigene Systeme ausgeprägt, um Themengebiete wie das Placement der Inhalte oder auch Nav-Meshes auf der Spielwelt abzudecken.


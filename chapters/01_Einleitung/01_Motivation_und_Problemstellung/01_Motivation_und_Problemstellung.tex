\section{Motivation und Problemstellung}

Die prozedurale Generierung von NPCs sowie das Erlernen von Bewegungen und das Animieren dieser stellt heutzutage in der Videospielindustrie eine zentrale Herausforderung dar. Nicht nur müssen Kreaturen effizient und somit oftmals zur Laufzeit generiert werden, sondern es müssen auch die Animationen welche durch maschinelles Lernen erlernt werden, auf ein möglichst breites Spektrum an Kreaturen anwendbar sein. Dabei ist es also essentiell, dass das Training der Kreaturen möglichst generalisiert stattfindet, sodass bei der Kreatur-Generierung eine große, sich bei den Körpermerkmalen variierende Menge der Kreaturen erzeugt werden kann, wodurch weniger Einschränkungen für die Körpereigenschaften der NPCs aufkommen. Somit sollten also die Animationen auf einen möglichst breiten Pool von Kreaturen anwendbar sein, ohne für zufällig erzeugte, potentiell neue, Körpermerkmale neu trainieren zu müssen.

Weiterhin muss bei der Generierung von Spielfiguren die relevante Herausforderung des automatischen Aufspannens eines Meshes über den bis zu diesem Zeitpunkt untexturierten Körper einer Kreatur gelöst werden; das sogenannte \textit{Automatic Rigging}, welches ebenfalls zu dem prozeduralen Erzeugen von Kreaturen dazugehört. Es muss also hier das Problem betrachtet werden, das Mesh welches ggf. ebenfalls prozedural erzeugt wird, auf das breite Spektrum von verschiedenen Körperausprägungen des Kreatur-Pools anwenden zu können, ohne dass es zu sehr von den Körperformen der Kreaturen abweicht, da sonst die Dreiecksnetze degenerieren und nachfolgende numerische Operationen fehlschlagen würden.

Ferner könnte die prozedurale Generierung von Inhalten in einem Videospiel nicht nur zum Erzeugen von Kreaturen zum Einsatz kommen, sondern auch für die Spielwelt selbst. Damit könnte ... \textcolor{red}{Notiz für Thomas: Sobald Inhalte von World-Generation stehen, hier ein wenig ausführen.}


  ... mesh muss zu creature passen... lernen der bewegung anpassbar bezüglich creature-pool
\section{Zielsetzung und Vorgehensweise}
\label{Zielsetzung_und_Vorgehensweise}

Im Rahmen der Projektgruppe 649 nehmen insgesamt 12 Teilnehmer an der Planung und Entwicklung des prozedural generierten 3D Rollenspiels teil. 

Für die Umsetzung des Projektes ist es vorgesehen die verschiedenen Bereiche des Rollenspiels aufzuteilen, sodass potentiell in Kleingruppen parallel an diesen gearbeitet werden kann. Diese Bereiche umfassen folgende Themengebiete: die Generierung von Spielleveln, die Evaluation der generierten Level, Generierung der Monster, Fortbewegung der Monster und zuletzt die Strategie bzw. Verhaltensweise der Monster. Innerhalb jeder dieser Bereiche sollen bestimmte Anforderungen, welche an die Gruppe vorgegeben sind, erfüllt sein: \newline \newline
\textbf{\textit{Die Generierung von Spielleveln}}\newline
\begin{itemize}
	\item Die Generierung der Spiellevel sollte prozedural durchgeführt werden: z.B. KD-Trees, L-Systeme, Space-Partitioning
	\item Die Spiellevel enthalten Böden, Wände und auch Spawn-Points für Objekte und NPCs
	\item Gleichzeitig sollen die Komplexität, Größe und der Schwierigkeitsgrad der Level parametrisierbar sein
\end{itemize}
\textbf{\textit{Evaluation der Generierten Level}}
\begin{itemize}
	\item Die Level/Spielwelten sollen anhand vorbestimmter Metriken evaluiert werden, wie z.B.: Lösbarkeit der Level, Schwierigkeitsgrad
	\item Folgende Ansätze sind dafür vorgeschlagen: Imitation Learning, Deep Reinforcement Learning
\end{itemize}
\textbf{\textit{Generierung der Monster}}
\begin{itemize}
	\item In diesem Bereich sind viele Freiheiten gelassen worden, da die Generierung der Monster auf viele unterschiedliche Weisen durchgeführt werden kann
	\item Relevant dabei ist nur, dass die Erstellung von NPCs algorithmus-basiert ist und dass innerhalb von Unity die von Unity bereitgestellten Joints verwendet werden sollten
	\item Eine Orientierungshilfe dabei kann der Unity-MLAgents-Walker sein
\end{itemize}
\textbf{\textit{Fortbewegung der Monster}}
\begin{itemize}
	\item Animationen sollen nicht händisch erstellt werden
	\item Mit Hilfe von Deep Reinforcement Learning können die Monster lernen sich zu bewegen
	\item Der Agent wählt die Kräfte aus, welche auf seine Joints ausgeübt werden
	\item Je nach lösen einer spezifischen Aufgabe wird der Agent dann belohnt
	\item Dabei könnten Inverse Kinematiken nützlich sein
\end{itemize}
\textbf{\textit{Strategie bzw. Verhaltensweise der Monster}}
\begin{itemize}
	\item Klassische Ansätze (high level) wären hier die Behavior Trees oder auch State machines
	\item Währenddessen lernende Ansätze (low level) wären Imitation Learning und Deep Reinforcement learning
\end{itemize}

Um ein grundlegendes Verständnis aller Teilnehmenden für alle Bereiche zu schaffen, werden im ersten Monat in einer Seminarphase alle Bereiche auf Kleingruppen, bestehend aus 3 Teilnehmern, aufgeteilt und potentielle Ansätze für die Umsetzung der Gebiete für ein Videospiel erforscht. Während dieser Seminarphase werden dann nacheinander die Themen der Gruppen den restlichen Teilnehmenden vorgestellt. Somit ist eine Verständnisgrundlage für alle Beteiligten geschaffen, sodass basierend auf dem allgemeinen Kenntnisstand mit der eigentlichen Entwicklung des Spieles angefangen werden kann. 

Ein besonderer Schwerpunkt soll dabei auf die Themen der \textbf{Generierung der Monster} und der \textbf{Fortbewegung der Monster} gelegt werden, da diese die Basis für die restlichen Aufgaben bilden sollten. Es wurde sich absichtlich dafür entschieden von Anfang an kein klares Design der späteren Spielwelt, Monster und des Spielercharakters zu definieren, sodass sich daraus keine Einschränkungen für die initiale Implementierungsphase ergeben sollten. Vielmehr sollte aus den Ergebnissen der initialen Phase das spätere Spieldesign abgeleitet werden. Damit also dieser Grundstein gelegt werden konnte, wurde sich dafür entschieden die \textbf{Generierung der Monster} und die \textbf{Fortbewegung der Monster} der Spielwelt als erstes zu priorisieren und somit die 12 Teilnehmer der Projektgruppe für die Anfangsphase in zwei Untergruppen aufzuteilen: die \textbf{Creature-Generator}, bestehend aus 7 Mitgliedern, und die \textbf{Creature-Animator}, bestehend aus 5 Mitgliedern. Es wird antizipiert, dass sich mit fortschreitender Zeit die Gruppen, sobald die Basis für das Spiel besteht, in weitere (kleinere) Gruppen aufteilen werden. Das Ziel ist somit, parallel an mehreren Themen gleichzeitig arbeiten zu können und somit effizient Fortschritt zu machen. Währenddessen muss eine deutliche Kommunikation zwischen den (Unter-)Gruppen bestehen um die Themen miteinander abstimmen zu können, vor allem für die Anfangsphase zwischen den \textbf{Creature-Generator} und den \textbf{Creature-Animator}.

Für diese Anfangsphase wurde evaluiert, die Kreatur-Generierung so einfach wie möglich zu halten um das Beibringen von Bewegungen durch die \textbf{Creature-Animator} Gruppe so einfach wie möglich zu gestalten und die Freiheitsgrade auf einem Minimum zu halten. Dafür ist das Vorhaben zweibeinige, humanoide Kreaturen und auch vierbeinige Kreaturen erzeugen zu können. Zusätzliche Körperteile wie beispielsweise Flügel, mehrere Arme oder Beine und Ähnliche Extras werden weggelassen, damit also keine zusätzlichen Freiheitsgrade hinzukommen und das Lernen der Bewegung der Kreatur gegebenenfalls erschweren würden. Trotzdem soll später evaluiert werden, ob solche ergänzenden Körperteile hinzugefügt werden könnten, sobald die Basisstruktur, also das stabile Erzeugen und Lernen der Bewegungen der Kreaturen, besteht. Außerdem wird durch diesen Ansatz der Spielentwurf so schlicht wie möglich gehalten, sodass keine potentiellen Einschränkungen bezüglich des Spieldesigns existieren und damit die spätere Gestaltung des Spiels basierend auf den dann bestehenden Funktionalitäten entschieden werden kann. 




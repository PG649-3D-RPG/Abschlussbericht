\section{Übersicht}

Während der Projektgruppe haben sich ... Untergruppen/Teams für die jeweiligen Bereiche ... herausgestellt.

Zuerst werden in dem zweiten Kapitel (\ref{Grundlagen}) alle Grundlagen beschrieben in Bezug auf die angeführten Themenbereiche der Projektgruppe welche in der Seminarphase gegenseitig vorgestellt wurden (\ref{Zielsetzung_und_Vorgehensweise}). Basierend auf diesen Themengebieten werden dann in Kapitel \ref{Verwandte_Arbeiten} verwandte Arbeiten und Literatur vorgestellt, welche sowohl während der Evaluation in der Seminarphase erforscht wurden, als auch in der späteren Erarbeitung weiterer Themen und Algorithmen (Kapitel \ref{Fachliches_Vorgehen} und \ref{Technische_Umsetzung}) ausgemacht wurden. Sobald die Grundlagen über die Themen der späteren fachlichen und technischen Vorgehensweise geklärt sind, wird in Kapitel \ref{Fachliches_Vorgehen} das fachliche Vorgehen beschrieben, welches sich während der Entwicklungsphase der verschiedenen Bereiche (s. Kapitel \ref{Zielsetzung_und_Vorgehensweise}) erschlossen hat. Nachdem die fachbezogene Vorgehensweise bekannt ist, wird im Anschluss die Umsetzung in Kapitel \ref{Technische_Umsetzung} evaluiert, wie die in Kapitel \ref{Fachliches_Vorgehen} beschriebenen Verfahren und Algorithmen technisch umgesetzt sind. Anschließend werden in Kapitel \ref{Vorlaeufige_Ergebnisse} die (\textcolor{red}{Zwischen-})Ergebnisse präsentiert und diskutiert (\ref{Diskussion}), \textcolor{red}{hier später für den Endbericht ausführen wie genau und wodrauf bezogen die Ergebnisse diskutiert werden}. Schließlich wird in Kapitel \ref{Ausblick} zusammengefasst, auf welche (Forschungs-)Bereiche die Erkenntnisse dieser Studie ausgeweitet werden können. \textcolor{red}{Weiter ausführen?}
%  Grundlagen L-System: Allgemein, Theoretischer Hintergrund, 2D, 3D
\section{L-System}
Eine zentrale Eigenschaft in der Natur ist die Selbstähnlichkeit von Strukturen auf makroskopischer und mikroskopischer Ebene~\cite{Shaker2016}. % ref this + reword
Hier lassen sich dieselben Strukturen in verschiedenen Größenordnungen wiederfinden. % delete this
Beispielsweise ähneln die Knospen eines Blumenkohls auch seiner äußeren Struktur, wie in Abbildung~\ref{fig:Blumenkohl}\footnote{Quelle: {https://commons.wikimedia.org/wiki/File:Blumenkohl-1.jpg}, Autor: Rainer Zenz, CC BY-SA 3.0 <http://creativecommons.org/licenses/by-sa/3.0/>, via Wikimedia Commons} zu sehen ist.
\begin{figure}[ht]
    \centering
    \includegraphics[width=0.5\linewidth]{chapters/02_Grundlagen/L_System/Blumenkohl-1.jpg}
    \caption{Blumenkohl}\label{fig:Blumenkohl}
\end{figure}

Um diese Eigenschaften präzise und strukturell darzustellen kann ein Lindenmayer-System (L-System)~\cite{lindenmayer1990} eingesetzt werden.
Die Basis eines L-Systems bildet eine kontextfreie Grammatik $G=(N,T,S,P)$ mit einer Menge von Nicht-Terminalsymbolen $N$, Terminalsymbolen $T$, einem Startsymbol $S$ und einer Menge von Produktionen $P$.
Der Unterschied zwischen einem L-System und einer üblichen kontextfreien Grammatik besteht darin, dass in einem Ableitungsschritt eines L-Systems parallel alle Nicht-Terminale ersetzt werden.
Zusätzlich werden nur eine feste Anzahl an Ableitungsschritten durchgeführt und es kann anstatt eines Startsymbols auch ein Startstring angegeben werden.

\subsection{Grafische Darstellung in 2D}
Der resultierende String einer Ableitung kann grafisch als Zeichenvorschriften für \emph{turtle graphics} gesehen werden. % reword
In \emph{turtle graphics} gibt es eine \emph{turtle}, oder Zeichenkopf, der initial in eine Richtung zeigt und sich in zwei oder drei Dimensionen fortbewegen kann.
Bei jeder Fortbewegung wird entlang der aktuellen Richtung des Kopfes und der hinterlegten Distanz eine Strecke gezeichnet.
Die Symbole der Grammatik werden hierbei als Fortbewegung um eine gewisse Distanz oder eine Drehung um eine der Achsen interpretiert.
In Abbildung~\ref{fig:L-System 2D Rotation} ist eine Visualisierung dieses Konzeptes zu sehen.
Hier ist der Zeichenkopf initial nach oben gerichtet.
% Add 2D rotation tikz
\begin{figure}[ht]
    \centering
    % \resizebox{\linewidth-3cm}{!}{%
        
    % }
    \includegraphics[width=0.5\linewidth]{example-image-a}
    \caption{L-System 2D Rotation}\label{fig:L-System 2D Rotation}
\end{figure}

Ohne Erweiterungen entstehen hierbei immer zusammenhängende Strukturen bei dem der Zeichenkopf nicht springen kann.
Das heißt, Blätter oder Äste sind nur schwer zu realisieren.


\subsection{Erweiterungen}
Damit komplexere Strukturen abgebildet werden können, werden zusätzliche Symbole eingesetzt und die Auswertung erweitert.
Zwei übliche Erweiterung die besonders zur Generierung von Vegetation nützlich sind, sind \emph{Bracketed L-Systeme} und \emph{Stochastische L-Systeme}.

\subsubsection{Bracketed L-Systeme}
Um Strukturen abzubilden die nur durch "absetzen" des Stiftes generiert werden können, können \emph{Bracketed L-Systeme}~\cite*{Shaker2016} eingesetzt werden.
Damit die generierte Struktur zusammenhängend bleibt, wird die Position und aktuelle Richtung des Zeichenkopfes auf einem Stack gespeichert und kann durch \texttt{push} und \texttt{pop} Operationen verwaltet werden.
Die Menge der Nicht-Terminalsymbole wird dabei um \texttt{[} für die \texttt{push} Operation und \texttt{]} für \texttt{pop} erweitert.

Mit dieser Erweiterung können nun auch Äste von Bäumen oder Stängel von Blumen gezeichnet werden.
Ein Beispiel für eine Art von Strauch kann mittels dieses L-Systems erzeugt werden:
\begin{itemize}
    \item Nicht-Terminalsymbole $N=\{F\}$
    \item Terminalsymbole $T=\{\texttt{+},\texttt{-},\texttt{[},\texttt{]}\}$
    \item Startstring $S=F$
    \item Produktionen:
    \begin{align*}
        F\rightarrow~F\texttt{[+}F\texttt{]}F\texttt{[-}F\texttt{]}\texttt{[}F\texttt{]}
    \end{align*}
\end{itemize}

Hierbei beschreibt das Nicht-Terminalsymbol $F$ eine Fortbewegung, \texttt{+} und \texttt{-} jeweils eine Drehung um 30 Grad nach links bzw. rechts in 2D und \texttt{[}, \texttt{]} beschreiben die \texttt{push} und \texttt{pop} Operationen.
Die generierten Strukturen sind in Abbildung~\ref{fig:Bracketed} zu sehen.

\begin{figure}[ht]
    \begin{subfigure}[t]{.3\textwidth}
        \centering
        \resizebox{!}{100px}{\begin{tikzpicture}\draw (0,0) -- (0,0);
\draw (0,0) -- (0,10);
\draw (0,10) -- (-5,19);
\draw (0,10) -- (0,10);
\draw (0,10) -- (0,20);
\draw (0,20) -- (5,29);
\draw (0,20) -- (0,20);
\draw (0,20) -- (0,30);
\draw (0,20) -- (0,20);
\end{tikzpicture}}

        \caption*{1 Ableitungsschritt}
    \end{subfigure}
    \hfill
    \begin{subfigure}[t]{.3\textwidth}
        \centering
        \resizebox{!}{100bp}{\begin{tikzpicture}\draw (0,0) -- (0,0);
\draw (0,0) -- (0,10);
\draw (0,10) -- (-5,19);
\draw (0,10) -- (0,10);
\draw (0,10) -- (0,20);
\draw (0,20) -- (5,29);
\draw (0,20) -- (0,20);
\draw (0,20) -- (0,30);
\draw (0,20) -- (0,20);
\draw (0,20) -- (-5,29);
\draw (-5,29) -- (-14,34);
\draw (-5,29) -- (-5,29);
\draw (-5,29) -- (-10,38);
\draw (-10,38) -- (-10,48);
\draw (-10,38) -- (-10,38);
\draw (-10,38) -- (-15,47);
\draw (-10,38) -- (-10,38);
\draw (0,20) -- (0,20);
\draw (0,20) -- (0,30);
\draw (0,30) -- (-5,39);
\draw (0,30) -- (0,30);
\draw (0,30) -- (0,40);
\draw (0,40) -- (5,49);
\draw (0,40) -- (0,40);
\draw (0,40) -- (0,50);
\draw (0,40) -- (0,40);
\draw (0,40) -- (5,49);
\draw (5,49) -- (5,59);
\draw (5,49) -- (5,49);
\draw (5,49) -- (10,58);
\draw (10,58) -- (19,63);
\draw (10,58) -- (10,58);
\draw (10,58) -- (15,67);
\draw (10,58) -- (10,58);
\draw (0,40) -- (0,40);
\draw (0,40) -- (0,50);
\draw (0,50) -- (-5,59);
\draw (0,50) -- (0,50);
\draw (0,50) -- (0,60);
\draw (0,60) -- (5,69);
\draw (0,60) -- (0,60);
\draw (0,60) -- (0,70);
\draw (0,60) -- (0,60);
\draw (0,40) -- (0,40);
\end{tikzpicture}}

        \caption*{2 Ableitungsschritte}
    \end{subfigure}
    \hfill
    \begin{subfigure}[t]{.3\textwidth}
        \centering
        \resizebox{!}{100bp}{\begin{tikzpicture}\draw (0,0) -- (0,0);
\draw (0,0) -- (0,10);
\draw (0,10) -- (-5,19);
\draw (0,10) -- (0,10);
\draw (0,10) -- (0,20);
\draw (0,20) -- (5,29);
\draw (0,20) -- (0,20);
\draw (0,20) -- (0,30);
\draw (0,20) -- (0,20);
\draw (0,20) -- (-5,29);
\draw (-5,29) -- (-14,34);
\draw (-5,29) -- (-5,29);
\draw (-5,29) -- (-10,38);
\draw (-10,38) -- (-10,48);
\draw (-10,38) -- (-10,38);
\draw (-10,38) -- (-15,47);
\draw (-10,38) -- (-10,38);
\draw (0,20) -- (0,20);
\draw (0,20) -- (0,30);
\draw (0,30) -- (-5,39);
\draw (0,30) -- (0,30);
\draw (0,30) -- (0,40);
\draw (0,40) -- (5,49);
\draw (0,40) -- (0,40);
\draw (0,40) -- (0,50);
\draw (0,40) -- (0,40);
\draw (0,40) -- (5,49);
\draw (5,49) -- (5,59);
\draw (5,49) -- (5,49);
\draw (5,49) -- (10,58);
\draw (10,58) -- (19,63);
\draw (10,58) -- (10,58);
\draw (10,58) -- (15,67);
\draw (10,58) -- (10,58);
\draw (0,40) -- (0,40);
\draw (0,40) -- (0,50);
\draw (0,50) -- (-5,59);
\draw (0,50) -- (0,50);
\draw (0,50) -- (0,60);
\draw (0,60) -- (5,69);
\draw (0,60) -- (0,60);
\draw (0,60) -- (0,70);
\draw (0,60) -- (0,60);
\draw (0,40) -- (0,40);
\draw (0,40) -- (-5,49);
\draw (-5,49) -- (-14,54);
\draw (-5,49) -- (-5,49);
\draw (-5,49) -- (-10,58);
\draw (-10,58) -- (-10,68);
\draw (-10,58) -- (-10,58);
\draw (-10,58) -- (-15,67);
\draw (-10,58) -- (-10,58);
\draw (-10,58) -- (-19,63);
\draw (-19,63) -- (-29,63);
\draw (-19,63) -- (-19,63);
\draw (-19,63) -- (-28,68);
\draw (-28,68) -- (-33,77);
\draw (-28,68) -- (-28,68);
\draw (-28,68) -- (-37,73);
\draw (-28,68) -- (-28,68);
\draw (-10,58) -- (-10,58);
\draw (-10,58) -- (-15,67);
\draw (-15,67) -- (-24,72);
\draw (-15,67) -- (-15,67);
\draw (-15,67) -- (-20,76);
\draw (-20,76) -- (-20,86);
\draw (-20,76) -- (-20,76);
\draw (-20,76) -- (-25,85);
\draw (-20,76) -- (-20,76);
\draw (-20,76) -- (-20,86);
\draw (-20,86) -- (-25,95);
\draw (-20,86) -- (-20,86);
\draw (-20,86) -- (-20,96);
\draw (-20,96) -- (-15,105);
\draw (-20,96) -- (-20,96);
\draw (-20,96) -- (-20,106);
\draw (-20,96) -- (-20,96);
\draw (-20,76) -- (-20,76);
\draw (-20,76) -- (-25,85);
\draw (-25,85) -- (-34,90);
\draw (-25,85) -- (-25,85);
\draw (-25,85) -- (-30,94);
\draw (-30,94) -- (-30,104);
\draw (-30,94) -- (-30,94);
\draw (-30,94) -- (-35,103);
\draw (-30,94) -- (-30,94);
\draw (-20,76) -- (-20,76);
\draw (0,40) -- (0,40);
\draw (0,40) -- (0,50);
\draw (0,50) -- (-5,59);
\draw (0,50) -- (0,50);
\draw (0,50) -- (0,60);
\draw (0,60) -- (5,69);
\draw (0,60) -- (0,60);
\draw (0,60) -- (0,70);
\draw (0,60) -- (0,60);
\draw (0,60) -- (-5,69);
\draw (-5,69) -- (-14,74);
\draw (-5,69) -- (-5,69);
\draw (-5,69) -- (-10,78);
\draw (-10,78) -- (-10,88);
\draw (-10,78) -- (-10,78);
\draw (-10,78) -- (-15,87);
\draw (-10,78) -- (-10,78);
\draw (0,60) -- (0,60);
\draw (0,60) -- (0,70);
\draw (0,70) -- (-5,79);
\draw (0,70) -- (0,70);
\draw (0,70) -- (0,80);
\draw (0,80) -- (5,89);
\draw (0,80) -- (0,80);
\draw (0,80) -- (0,90);
\draw (0,80) -- (0,80);
\draw (0,80) -- (5,89);
\draw (5,89) -- (5,99);
\draw (5,89) -- (5,89);
\draw (5,89) -- (10,98);
\draw (10,98) -- (19,103);
\draw (10,98) -- (10,98);
\draw (10,98) -- (15,107);
\draw (10,98) -- (10,98);
\draw (0,80) -- (0,80);
\draw (0,80) -- (0,90);
\draw (0,90) -- (-5,99);
\draw (0,90) -- (0,90);
\draw (0,90) -- (0,100);
\draw (0,100) -- (5,109);
\draw (0,100) -- (0,100);
\draw (0,100) -- (0,110);
\draw (0,100) -- (0,100);
\draw (0,80) -- (0,80);
\draw (0,80) -- (5,89);
\draw (5,89) -- (5,99);
\draw (5,89) -- (5,89);
\draw (5,89) -- (10,98);
\draw (10,98) -- (19,103);
\draw (10,98) -- (10,98);
\draw (10,98) -- (15,107);
\draw (10,98) -- (10,98);
\draw (10,98) -- (10,108);
\draw (10,108) -- (5,117);
\draw (10,108) -- (10,108);
\draw (10,108) -- (10,118);
\draw (10,118) -- (15,127);
\draw (10,118) -- (10,118);
\draw (10,118) -- (10,128);
\draw (10,118) -- (10,118);
\draw (10,98) -- (10,98);
\draw (10,98) -- (15,107);
\draw (15,107) -- (15,117);
\draw (15,107) -- (15,107);
\draw (15,107) -- (20,116);
\draw (20,116) -- (29,121);
\draw (20,116) -- (20,116);
\draw (20,116) -- (25,125);
\draw (20,116) -- (20,116);
\draw (20,116) -- (29,121);
\draw (29,121) -- (34,130);
\draw (29,121) -- (29,121);
\draw (29,121) -- (38,126);
\draw (38,126) -- (48,126);
\draw (38,126) -- (38,126);
\draw (38,126) -- (47,131);
\draw (38,126) -- (38,126);
\draw (20,116) -- (20,116);
\draw (20,116) -- (25,125);
\draw (25,125) -- (25,135);
\draw (25,125) -- (25,125);
\draw (25,125) -- (30,134);
\draw (30,134) -- (39,139);
\draw (30,134) -- (30,134);
\draw (30,134) -- (35,143);
\draw (30,134) -- (30,134);
\draw (20,116) -- (20,116);
\draw (0,80) -- (0,80);
\draw (0,80) -- (0,90);
\draw (0,90) -- (-5,99);
\draw (0,90) -- (0,90);
\draw (0,90) -- (0,100);
\draw (0,100) -- (5,109);
\draw (0,100) -- (0,100);
\draw (0,100) -- (0,110);
\draw (0,100) -- (0,100);
\draw (0,100) -- (-5,109);
\draw (-5,109) -- (-14,114);
\draw (-5,109) -- (-5,109);
\draw (-5,109) -- (-10,118);
\draw (-10,118) -- (-10,128);
\draw (-10,118) -- (-10,118);
\draw (-10,118) -- (-15,127);
\draw (-10,118) -- (-10,118);
\draw (0,100) -- (0,100);
\draw (0,100) -- (0,110);
\draw (0,110) -- (-5,119);
\draw (0,110) -- (0,110);
\draw (0,110) -- (0,120);
\draw (0,120) -- (5,129);
\draw (0,120) -- (0,120);
\draw (0,120) -- (0,130);
\draw (0,120) -- (0,120);
\draw (0,120) -- (5,129);
\draw (5,129) -- (5,139);
\draw (5,129) -- (5,129);
\draw (5,129) -- (10,138);
\draw (10,138) -- (19,143);
\draw (10,138) -- (10,138);
\draw (10,138) -- (15,147);
\draw (10,138) -- (10,138);
\draw (0,120) -- (0,120);
\draw (0,120) -- (0,130);
\draw (0,130) -- (-5,139);
\draw (0,130) -- (0,130);
\draw (0,130) -- (0,140);
\draw (0,140) -- (5,149);
\draw (0,140) -- (0,140);
\draw (0,140) -- (0,150);
\draw (0,140) -- (0,140);
\draw (0,120) -- (0,120);
\draw (0,80) -- (0,80);
\end{tikzpicture}}

        \caption*{3 Ableitungsschritte}
    \end{subfigure}
    \caption{Resultate bracketed L-System}\label{fig:Bracketed}
\end{figure}


\subsubsection{Stochastische L-Systeme}
Eine Eigenschaft von den bisher verwendeten L-Systemen ist, dass die Produktionen deterministisch ausgewertet werden.
Um realistische Pflanzen generieren zu können muss jedoch etwas Variation eingeführt werden; es wäre also vorteilhaft, wenn die Produktionen nicht-deterministisch ausgewertet werden.
Dazu kann ein \emph{stochastisches L-System}~\cite*{Shaker2016} verwendet werden.
Hierbei kann es mehrere Produktionen mit gleichen linken Seiten geben.
Jeweils über die Menge von Produktionen mit gleicher linken Seite wird eine Wahrscheinlichkeitsverteilung erstellt.
Das heißt die Auswahl der Ersetzung eines Nicht-Terminalsymbols erfolgt zufällig.

Im Folgenden ist ein Beispiel für ein stochastisches bracketed L-System gegeben:
\begin{itemize}
    \item Nicht-Terminalsymbole $N=\{F\}$
    \item Terminalsymbole $T=\{\texttt{+},\texttt{-},\texttt{[},\texttt{]}\}$
    \item Startstring $S=F$
    \item Produktionen:
    \begin{align*}
        F\xrightarrow{0.33} & ~F\texttt{[+}F\texttt{]}F\texttt{[-}F\texttt{]} F \\
        F\xrightarrow{0.33} & ~F\texttt{[+}F\texttt{]} F                         \\
        F\xrightarrow{0.34} & ~F\texttt{[-}F\texttt{]} F
    \end{align*}
\end{itemize}

Hier gibt es für das Nicht-Terminal $F$ drei Produktionen, zwei Produktionen mit einer Wahrscheinlichkeit von $33\%$ gewählt und eine Produktion mit $34\%$.
Drei resultierende Bilder dieses L-Systems sind in Abbildung~\ref{fig:Stochastic} gegeben.
Hieran kann man gut sehen wie aus einem L-System deutlich unterschiedliche Strukturen generiert werden können die jedoch alle ähnliche Grundstrukturen aufweisen.
\begin{figure}[ht]
    \begin{subfigure}[t]{.25\textwidth}
        \centering
        \resizebox{!}{175px}{\begin{tikzpicture}\draw (0,0) -- (0,0);
\draw (0,0) -- (0,10);
\draw (0,10) -- (-5,19);
\draw (0,10) -- (0,10);
\draw (0,10) -- (0,20);
\draw (0,20) -- (5,29);
\draw (5,29) -- (5,39);
\draw (5,29) -- (5,29);
\draw (5,29) -- (10,38);
\draw (0,20) -- (0,20);
\draw (0,20) -- (0,30);
\draw (0,30) -- (-5,39);
\draw (0,30) -- (0,30);
\draw (0,30) -- (0,40);
\draw (0,40) -- (5,49);
\draw (5,49) -- (14,54);
\draw (5,49) -- (5,49);
\draw (5,49) -- (10,58);
\draw (10,58) -- (10,68);
\draw (10,68) -- (5,77);
\draw (10,68) -- (10,68);
\draw (10,68) -- (10,78);
\draw (10,58) -- (10,58);
\draw (10,58) -- (15,67);
\draw (15,67) -- (15,77);
\draw (15,67) -- (15,67);
\draw (15,67) -- (20,76);
\draw (0,40) -- (0,40);
\draw (0,40) -- (0,50);
\draw (0,50) -- (5,59);
\draw (0,50) -- (0,50);
\draw (0,50) -- (0,60);
\draw (0,60) -- (-5,69);
\draw (0,60) -- (0,60);
\draw (0,60) -- (0,70);
\draw (0,70) -- (5,79);
\draw (5,79) -- (5,89);
\draw (5,79) -- (5,79);
\draw (5,79) -- (10,88);
\draw (0,70) -- (0,70);
\draw (0,70) -- (0,80);
\draw (0,80) -- (5,89);
\draw (0,80) -- (0,80);
\draw (0,80) -- (0,90);
\draw (0,90) -- (-5,99);
\draw (0,90) -- (0,90);
\draw (0,90) -- (0,100);
\draw (0,100) -- (-5,109);
\draw (-5,109) -- (-5,119);
\draw (-5,109) -- (-5,109);
\draw (-5,109) -- (-10,118);
\draw (-10,118) -- (-19,123);
\draw (-10,118) -- (-10,118);
\draw (-10,118) -- (-15,127);
\draw (-15,127) -- (-15,137);
\draw (-15,137) -- (-10,146);
\draw (-15,137) -- (-15,137);
\draw (-15,137) -- (-15,147);
\draw (-15,127) -- (-15,127);
\draw (-15,127) -- (-20,136);
\draw (-20,136) -- (-20,146);
\draw (-20,136) -- (-20,136);
\draw (-20,136) -- (-25,145);
\draw (-25,145) -- (-34,150);
\draw (-34,150) -- (-44,150);
\draw (-34,150) -- (-34,150);
\draw (-34,150) -- (-43,155);
\draw (-25,145) -- (-25,145);
\draw (-25,145) -- (-30,154);
\draw (-30,154) -- (-30,164);
\draw (-30,154) -- (-30,154);
\draw (-30,154) -- (-35,163);
\draw (-35,163) -- (-44,168);
\draw (-35,163) -- (-35,163);
\draw (-35,163) -- (-40,172);
\draw (0,100) -- (0,100);
\draw (0,100) -- (0,110);
\draw (0,110) -- (-5,119);
\draw (0,110) -- (0,110);
\draw (0,110) -- (0,120);
\draw (0,120) -- (-5,129);
\draw (-5,129) -- (-5,139);
\draw (-5,129) -- (-5,129);
\draw (-5,129) -- (-10,138);
\draw (-10,138) -- (-19,143);
\draw (-10,138) -- (-10,138);
\draw (-10,138) -- (-15,147);
\draw (0,120) -- (0,120);
\draw (0,120) -- (0,130);
\draw (0,130) -- (5,139);
\draw (0,130) -- (0,130);
\draw (0,130) -- (0,140);
\draw (0,140) -- (-5,149);
\draw (0,140) -- (0,140);
\draw (0,140) -- (0,150);
\draw (0,150) -- (5,159);
\draw (5,159) -- (14,164);
\draw (5,159) -- (5,159);
\draw (5,159) -- (10,168);
\draw (10,168) -- (10,178);
\draw (10,168) -- (10,168);
\draw (10,168) -- (15,177);
\draw (15,177) -- (24,182);
\draw (24,182) -- (29,191);
\draw (24,182) -- (24,182);
\draw (24,182) -- (33,187);
\draw (15,177) -- (15,177);
\draw (15,177) -- (20,186);
\draw (20,186) -- (29,191);
\draw (20,186) -- (20,186);
\draw (20,186) -- (25,195);
\draw (25,195) -- (25,205);
\draw (25,205) -- (30,214);
\draw (25,205) -- (25,205);
\draw (25,205) -- (25,215);
\draw (25,215) -- (20,224);
\draw (25,215) -- (25,215);
\draw (25,215) -- (25,225);
\draw (25,225) -- (30,234);
\draw (30,234) -- (39,239);
\draw (30,234) -- (30,234);
\draw (30,234) -- (35,243);
\draw (35,243) -- (35,253);
\draw (35,243) -- (35,243);
\draw (35,243) -- (40,252);
\draw (25,225) -- (25,225);
\draw (25,225) -- (25,235);
\draw (25,235) -- (20,244);
\draw (25,235) -- (25,235);
\draw (25,235) -- (25,245);
\draw (25,245) -- (20,254);
\draw (20,254) -- (20,264);
\draw (20,254) -- (20,254);
\draw (20,254) -- (15,263);
\draw (25,245) -- (25,245);
\draw (25,245) -- (25,255);
\draw (25,255) -- (30,264);
\draw (25,255) -- (25,255);
\draw (25,255) -- (25,265);
\draw (25,265) -- (20,274);
\draw (25,265) -- (25,265);
\draw (25,265) -- (25,275);
\draw (25,195) -- (25,195);
\draw (25,195) -- (30,204);
\draw (30,204) -- (30,214);
\draw (30,204) -- (30,204);
\draw (30,204) -- (35,213);
\draw (35,213) -- (35,223);
\draw (35,223) -- (30,232);
\draw (35,223) -- (35,223);
\draw (35,223) -- (35,233);
\draw (35,213) -- (35,213);
\draw (35,213) -- (40,222);
\draw (40,222) -- (49,227);
\draw (40,222) -- (40,222);
\draw (40,222) -- (45,231);
\draw (45,231) -- (45,241);
\draw (45,231) -- (45,231);
\draw (45,231) -- (50,240);
\draw (0,150) -- (0,150);
\draw (0,150) -- (0,160);
\draw (0,160) -- (-5,169);
\draw (0,160) -- (0,160);
\draw (0,160) -- (0,170);
\draw (0,170) -- (5,179);
\draw (5,179) -- (14,184);
\draw (5,179) -- (5,179);
\draw (5,179) -- (10,188);
\draw (10,188) -- (10,198);
\draw (10,188) -- (10,188);
\draw (10,188) -- (15,197);
\draw (0,170) -- (0,170);
\draw (0,170) -- (0,180);
\draw (0,180) -- (-5,189);
\draw (0,180) -- (0,180);
\draw (0,180) -- (0,190);
\draw (0,190) -- (5,199);
\draw (5,199) -- (14,204);
\draw (5,199) -- (5,199);
\draw (5,199) -- (10,208);
\draw (10,208) -- (10,218);
\draw (10,208) -- (10,208);
\draw (10,208) -- (15,217);
\draw (15,217) -- (24,222);
\draw (24,222) -- (29,231);
\draw (24,222) -- (24,222);
\draw (24,222) -- (33,227);
\draw (15,217) -- (15,217);
\draw (15,217) -- (20,226);
\draw (20,226) -- (29,231);
\draw (20,226) -- (20,226);
\draw (20,226) -- (25,235);
\draw (25,235) -- (25,245);
\draw (25,245) -- (30,254);
\draw (25,245) -- (25,245);
\draw (25,245) -- (25,255);
\draw (25,235) -- (25,235);
\draw (25,235) -- (30,244);
\draw (30,244) -- (39,249);
\draw (30,244) -- (30,244);
\draw (30,244) -- (35,253);
\draw (0,190) -- (0,190);
\draw (0,190) -- (0,200);
\draw (0,200) -- (5,209);
\draw (0,200) -- (0,200);
\draw (0,200) -- (0,210);
\draw (0,210) -- (-5,219);
\draw (0,210) -- (0,210);
\draw (0,210) -- (0,220);
\draw (0,220) -- (5,229);
\draw (5,229) -- (14,234);
\draw (5,229) -- (5,229);
\draw (5,229) -- (10,238);
\draw (0,220) -- (0,220);
\draw (0,220) -- (0,230);
\draw (0,230) -- (5,239);
\draw (0,230) -- (0,230);
\draw (0,230) -- (0,240);
\draw (0,240) -- (-5,249);
\draw (0,240) -- (0,240);
\draw (0,240) -- (0,250);
\draw (0,250) -- (-5,259);
\draw (-5,259) -- (-14,264);
\draw (-5,259) -- (-5,259);
\draw (-5,259) -- (-10,268);
\draw (-10,268) -- (-19,273);
\draw (-19,273) -- (-24,282);
\draw (-19,273) -- (-19,273);
\draw (-19,273) -- (-28,278);
\draw (-10,268) -- (-10,268);
\draw (-10,268) -- (-15,277);
\draw (-15,277) -- (-15,287);
\draw (-15,277) -- (-15,277);
\draw (-15,277) -- (-20,286);
\draw (-20,286) -- (-29,291);
\draw (-20,286) -- (-20,286);
\draw (-20,286) -- (-25,295);
\draw (-25,295) -- (-25,305);
\draw (-25,305) -- (-20,314);
\draw (-25,305) -- (-25,305);
\draw (-25,305) -- (-25,315);
\draw (-25,315) -- (-30,324);
\draw (-25,315) -- (-25,315);
\draw (-25,315) -- (-25,325);
\draw (-25,325) -- (-30,334);
\draw (-30,334) -- (-30,344);
\draw (-30,334) -- (-30,334);
\draw (-30,334) -- (-35,343);
\draw (-25,325) -- (-25,325);
\draw (-25,325) -- (-25,335);
\draw (-25,335) -- (-30,344);
\draw (-25,335) -- (-25,335);
\draw (-25,335) -- (-25,345);
\draw (-25,295) -- (-25,295);
\draw (-25,295) -- (-30,304);
\draw (-30,304) -- (-30,314);
\draw (-30,304) -- (-30,304);
\draw (-30,304) -- (-35,313);
\draw (-35,313) -- (-44,318);
\draw (-35,313) -- (-35,313);
\draw (-35,313) -- (-40,322);
\draw (-40,322) -- (-40,332);
\draw (-40,332) -- (-35,341);
\draw (-40,332) -- (-40,332);
\draw (-40,332) -- (-40,342);
\draw (-40,322) -- (-40,322);
\draw (-40,322) -- (-45,331);
\draw (-45,331) -- (-45,341);
\draw (-45,331) -- (-45,331);
\draw (-45,331) -- (-50,340);
\draw (-50,340) -- (-59,345);
\draw (-50,340) -- (-50,340);
\draw (-50,340) -- (-55,349);
\draw (-55,349) -- (-64,354);
\draw (-64,354) -- (-69,363);
\draw (-64,354) -- (-64,354);
\draw (-64,354) -- (-73,359);
\draw (-73,359) -- (-83,359);
\draw (-73,359) -- (-73,359);
\draw (-73,359) -- (-82,364);
\draw (-55,349) -- (-55,349);
\draw (-55,349) -- (-60,358);
\draw (-60,358) -- (-69,363);
\draw (-60,358) -- (-60,358);
\draw (-60,358) -- (-65,367);
\draw (0,250) -- (0,250);
\draw (0,250) -- (0,260);
\draw (0,260) -- (5,269);
\draw (0,260) -- (0,260);
\draw (0,260) -- (0,270);
\draw (0,270) -- (-5,279);
\draw (-5,279) -- (-5,289);
\draw (-5,279) -- (-5,279);
\draw (-5,279) -- (-10,288);
\draw (0,270) -- (0,270);
\draw (0,270) -- (0,280);
\draw (0,280) -- (5,289);
\draw (0,280) -- (0,280);
\draw (0,280) -- (0,290);
\draw (0,290) -- (-5,299);
\draw (0,290) -- (0,290);
\draw (0,290) -- (0,300);
\draw (0,300) -- (5,309);
\draw (5,309) -- (5,319);
\draw (5,309) -- (5,309);
\draw (5,309) -- (10,318);
\draw (10,318) -- (19,323);
\draw (19,323) -- (24,332);
\draw (19,323) -- (19,323);
\draw (19,323) -- (28,328);
\draw (10,318) -- (10,318);
\draw (10,318) -- (15,327);
\draw (15,327) -- (15,337);
\draw (15,327) -- (15,327);
\draw (15,327) -- (20,336);
\draw (20,336) -- (20,346);
\draw (20,346) -- (15,355);
\draw (20,346) -- (20,346);
\draw (20,346) -- (20,356);
\draw (20,336) -- (20,336);
\draw (20,336) -- (25,345);
\draw (25,345) -- (34,350);
\draw (25,345) -- (25,345);
\draw (25,345) -- (30,354);
\draw (30,354) -- (30,364);
\draw (30,354) -- (30,354);
\draw (30,354) -- (35,363);
\draw (0,300) -- (0,300);
\draw (0,300) -- (0,310);
\draw (0,310) -- (-5,319);
\draw (0,310) -- (0,310);
\draw (0,310) -- (0,320);
\draw (0,320) -- (5,329);
\draw (5,329) -- (14,334);
\draw (5,329) -- (5,329);
\draw (5,329) -- (10,338);
\draw (10,338) -- (10,348);
\draw (10,338) -- (10,338);
\draw (10,338) -- (15,347);
\draw (0,320) -- (0,320);
\draw (0,320) -- (0,330);
\draw (0,330) -- (5,339);
\draw (0,330) -- (0,330);
\draw (0,330) -- (0,340);
\draw (0,340) -- (-5,349);
\draw (-5,349) -- (-14,354);
\draw (-5,349) -- (-5,349);
\draw (-5,349) -- (-10,358);
\draw (-10,358) -- (-10,368);
\draw (-10,368) -- (-5,377);
\draw (-10,368) -- (-10,368);
\draw (-10,368) -- (-10,378);
\draw (-10,358) -- (-10,358);
\draw (-10,358) -- (-15,367);
\draw (-15,367) -- (-15,377);
\draw (-15,367) -- (-15,367);
\draw (-15,367) -- (-20,376);
\draw (-20,376) -- (-29,381);
\draw (-20,376) -- (-20,376);
\draw (-20,376) -- (-25,385);
\draw (0,340) -- (0,340);
\draw (0,340) -- (0,350);
\draw (0,350) -- (-5,359);
\draw (0,350) -- (0,350);
\draw (0,350) -- (0,360);
\draw (0,360) -- (5,369);
\draw (5,369) -- (14,374);
\draw (5,369) -- (5,369);
\draw (5,369) -- (10,378);
\draw (0,360) -- (0,360);
\draw (0,360) -- (0,370);
\draw (0,370) -- (-5,379);
\draw (0,370) -- (0,370);
\draw (0,370) -- (0,380);
\draw (0,380) -- (-5,389);
\draw (-5,389) -- (-14,394);
\draw (-5,389) -- (-5,389);
\draw (-5,389) -- (-10,398);
\draw (0,380) -- (0,380);
\draw (0,380) -- (0,390);
\draw (0,390) -- (5,399);
\draw (0,390) -- (0,390);
\draw (0,390) -- (0,400);
\draw (0,400) -- (-5,409);
\draw (0,400) -- (0,400);
\draw (0,400) -- (0,410);
\end{tikzpicture}}

    \end{subfigure}
    \hfill
    \begin{subfigure}[t]{.25\textwidth}
        \centering
        \resizebox{!}{175bp}{\begin{tikzpicture}\draw (0,0) -- (0,0);
\draw (0,0) -- (0,10);
\draw (0,10) -- (5,19);
\draw (0,10) -- (0,10);
\draw (0,10) -- (0,20);
\draw (0,20) -- (5,29);
\draw (5,29) -- (5,39);
\draw (5,29) -- (5,29);
\draw (5,29) -- (10,38);
\draw (0,20) -- (0,20);
\draw (0,20) -- (0,30);
\draw (0,30) -- (5,39);
\draw (0,30) -- (0,30);
\draw (0,30) -- (0,40);
\draw (0,40) -- (-5,49);
\draw (0,40) -- (0,40);
\draw (0,40) -- (0,50);
\draw (0,50) -- (-5,59);
\draw (-5,59) -- (-5,69);
\draw (-5,59) -- (-5,59);
\draw (-5,59) -- (-10,68);
\draw (-10,68) -- (-19,73);
\draw (-10,68) -- (-10,68);
\draw (-10,68) -- (-15,77);
\draw (0,50) -- (0,50);
\draw (0,50) -- (0,60);
\draw (0,60) -- (5,69);
\draw (0,60) -- (0,60);
\draw (0,60) -- (0,70);
\draw (0,70) -- (-5,79);
\draw (-5,79) -- (-14,84);
\draw (-5,79) -- (-5,79);
\draw (-5,79) -- (-10,88);
\draw (-10,88) -- (-10,98);
\draw (-10,98) -- (-15,107);
\draw (-10,98) -- (-10,98);
\draw (-10,98) -- (-10,108);
\draw (-10,88) -- (-10,88);
\draw (-10,88) -- (-15,97);
\draw (-15,97) -- (-24,102);
\draw (-15,97) -- (-15,97);
\draw (-15,97) -- (-20,106);
\draw (-20,106) -- (-29,111);
\draw (-29,111) -- (-39,111);
\draw (-29,111) -- (-29,111);
\draw (-29,111) -- (-38,116);
\draw (-20,106) -- (-20,106);
\draw (-20,106) -- (-25,115);
\draw (-25,115) -- (-34,120);
\draw (-25,115) -- (-25,115);
\draw (-25,115) -- (-30,124);
\draw (0,70) -- (0,70);
\draw (0,70) -- (0,80);
\draw (0,80) -- (5,89);
\draw (0,80) -- (0,80);
\draw (0,80) -- (0,90);
\draw (0,90) -- (-5,99);
\draw (0,90) -- (0,90);
\draw (0,90) -- (0,100);
\draw (0,100) -- (5,109);
\draw (5,109) -- (5,119);
\draw (5,109) -- (5,109);
\draw (5,109) -- (10,118);
\draw (0,100) -- (0,100);
\draw (0,100) -- (0,110);
\draw (0,110) -- (-5,119);
\draw (0,110) -- (0,110);
\draw (0,110) -- (0,120);
\draw (0,120) -- (-5,129);
\draw (-5,129) -- (-5,139);
\draw (-5,129) -- (-5,129);
\draw (-5,129) -- (-10,138);
\draw (-10,138) -- (-19,143);
\draw (-10,138) -- (-10,138);
\draw (-10,138) -- (-15,147);
\draw (-15,147) -- (-15,157);
\draw (-15,157) -- (-10,166);
\draw (-15,157) -- (-15,157);
\draw (-15,157) -- (-15,167);
\draw (-15,147) -- (-15,147);
\draw (-15,147) -- (-20,156);
\draw (-20,156) -- (-20,166);
\draw (-20,156) -- (-20,156);
\draw (-20,156) -- (-25,165);
\draw (-25,165) -- (-34,170);
\draw (-25,165) -- (-25,165);
\draw (-25,165) -- (-30,174);
\draw (-30,174) -- (-30,184);
\draw (-30,184) -- (-25,193);
\draw (-30,184) -- (-30,184);
\draw (-30,184) -- (-30,194);
\draw (-30,194) -- (-25,203);
\draw (-25,203) -- (-25,213);
\draw (-25,203) -- (-25,203);
\draw (-25,203) -- (-20,212);
\draw (-30,194) -- (-30,194);
\draw (-30,194) -- (-30,204);
\draw (-30,204) -- (-25,213);
\draw (-30,204) -- (-30,204);
\draw (-30,204) -- (-30,214);
\draw (-30,214) -- (-35,223);
\draw (-35,223) -- (-35,233);
\draw (-35,223) -- (-35,223);
\draw (-35,223) -- (-40,232);
\draw (-40,232) -- (-49,237);
\draw (-40,232) -- (-40,232);
\draw (-40,232) -- (-45,241);
\draw (-30,214) -- (-30,214);
\draw (-30,214) -- (-30,224);
\draw (-30,224) -- (-25,233);
\draw (-30,224) -- (-30,224);
\draw (-30,224) -- (-30,234);
\draw (-30,234) -- (-35,243);
\draw (-30,234) -- (-30,234);
\draw (-30,234) -- (-30,244);
\draw (-30,174) -- (-30,174);
\draw (-30,174) -- (-35,183);
\draw (-35,183) -- (-35,193);
\draw (-35,183) -- (-35,183);
\draw (-35,183) -- (-40,192);
\draw (-40,192) -- (-49,197);
\draw (-40,192) -- (-40,192);
\draw (-40,192) -- (-45,201);
\draw (-45,201) -- (-54,206);
\draw (-54,206) -- (-64,206);
\draw (-54,206) -- (-54,206);
\draw (-54,206) -- (-63,211);
\draw (-45,201) -- (-45,201);
\draw (-45,201) -- (-50,210);
\draw (-50,210) -- (-50,220);
\draw (-50,210) -- (-50,210);
\draw (-50,210) -- (-55,219);
\draw (-55,219) -- (-64,224);
\draw (-55,219) -- (-55,219);
\draw (-55,219) -- (-60,228);
\draw (-60,228) -- (-69,233);
\draw (-69,233) -- (-79,233);
\draw (-69,233) -- (-69,233);
\draw (-69,233) -- (-78,238);
\draw (-78,238) -- (-83,247);
\draw (-83,247) -- (-83,257);
\draw (-83,247) -- (-83,247);
\draw (-83,247) -- (-88,256);
\draw (-78,238) -- (-78,238);
\draw (-78,238) -- (-87,243);
\draw (-87,243) -- (-97,243);
\draw (-87,243) -- (-87,243);
\draw (-87,243) -- (-96,248);
\draw (-60,228) -- (-60,228);
\draw (-60,228) -- (-65,237);
\draw (-65,237) -- (-74,242);
\draw (-65,237) -- (-65,237);
\draw (-65,237) -- (-70,246);
\draw (-70,246) -- (-70,256);
\draw (-70,256) -- (-65,265);
\draw (-70,256) -- (-70,256);
\draw (-70,256) -- (-70,266);
\draw (-70,246) -- (-70,246);
\draw (-70,246) -- (-75,255);
\draw (-75,255) -- (-84,260);
\draw (-75,255) -- (-75,255);
\draw (-75,255) -- (-80,264);
\draw (-80,264) -- (-89,269);
\draw (-89,269) -- (-94,278);
\draw (-89,269) -- (-89,269);
\draw (-89,269) -- (-98,274);
\draw (-98,274) -- (-108,274);
\draw (-98,274) -- (-98,274);
\draw (-98,274) -- (-107,279);
\draw (-80,264) -- (-80,264);
\draw (-80,264) -- (-85,273);
\draw (-85,273) -- (-85,283);
\draw (-85,273) -- (-85,273);
\draw (-85,273) -- (-90,282);
\draw (-90,282) -- (-99,287);
\draw (-90,282) -- (-90,282);
\draw (-90,282) -- (-95,291);
\draw (0,120) -- (0,120);
\draw (0,120) -- (0,130);
\draw (0,130) -- (5,139);
\draw (0,130) -- (0,130);
\draw (0,130) -- (0,140);
\draw (0,140) -- (-5,149);
\draw (0,140) -- (0,140);
\draw (0,140) -- (0,150);
\draw (0,150) -- (-5,159);
\draw (-5,159) -- (-14,164);
\draw (-5,159) -- (-5,159);
\draw (-5,159) -- (-10,168);
\draw (0,150) -- (0,150);
\draw (0,150) -- (0,160);
\draw (0,160) -- (5,169);
\draw (0,160) -- (0,160);
\draw (0,160) -- (0,170);
\draw (0,170) -- (-5,179);
\draw (-5,179) -- (-5,189);
\draw (-5,179) -- (-5,179);
\draw (-5,179) -- (-10,188);
\draw (-10,188) -- (-19,193);
\draw (-10,188) -- (-10,188);
\draw (-10,188) -- (-15,197);
\draw (-15,197) -- (-15,207);
\draw (-15,207) -- (-20,216);
\draw (-15,207) -- (-15,207);
\draw (-15,207) -- (-15,217);
\draw (-15,197) -- (-15,197);
\draw (-15,197) -- (-20,206);
\draw (-20,206) -- (-29,211);
\draw (-20,206) -- (-20,206);
\draw (-20,206) -- (-25,215);
\draw (0,170) -- (0,170);
\draw (0,170) -- (0,180);
\draw (0,180) -- (-5,189);
\draw (0,180) -- (0,180);
\draw (0,180) -- (0,190);
\draw (0,190) -- (-5,199);
\draw (-5,199) -- (-5,209);
\draw (-5,199) -- (-5,199);
\draw (-5,199) -- (-10,208);
\draw (0,190) -- (0,190);
\draw (0,190) -- (0,200);
\draw (0,200) -- (5,209);
\draw (0,200) -- (0,200);
\draw (0,200) -- (0,210);
\draw (0,210) -- (-5,219);
\draw (0,210) -- (0,210);
\draw (0,210) -- (0,220);
\end{tikzpicture}}

    \end{subfigure}
    \hfill
    \begin{subfigure}[t]{.25\textwidth}
        \centering
        \resizebox{!}{175bp}{\begin{tikzpicture}\draw (0,0) -- (0,0);
\draw (0,0) -- (0,10);
\draw (0,10) -- (5,19);
\draw (0,10) -- (0,10);
\draw (0,10) -- (0,20);
\draw (0,20) -- (-5,29);
\draw (0,20) -- (0,20);
\draw (0,20) -- (0,30);
\draw (0,30) -- (-5,39);
\draw (-5,39) -- (-14,44);
\draw (-5,39) -- (-5,39);
\draw (-5,39) -- (-10,48);
\draw (0,30) -- (0,30);
\draw (0,30) -- (0,40);
\draw (0,40) -- (-5,49);
\draw (0,40) -- (0,40);
\draw (0,40) -- (0,50);
\draw (0,50) -- (-5,59);
\draw (-5,59) -- (-5,69);
\draw (-5,59) -- (-5,59);
\draw (-5,59) -- (-10,68);
\draw (-10,68) -- (-19,73);
\draw (-19,73) -- (-29,73);
\draw (-19,73) -- (-19,73);
\draw (-19,73) -- (-28,78);
\draw (-10,68) -- (-10,68);
\draw (-10,68) -- (-15,77);
\draw (-15,77) -- (-15,87);
\draw (-15,77) -- (-15,77);
\draw (-15,77) -- (-20,86);
\draw (-20,86) -- (-29,91);
\draw (-20,86) -- (-20,86);
\draw (-20,86) -- (-25,95);
\draw (0,50) -- (0,50);
\draw (0,50) -- (0,60);
\draw (0,60) -- (5,69);
\draw (0,60) -- (0,60);
\draw (0,60) -- (0,70);
\draw (0,70) -- (-5,79);
\draw (0,70) -- (0,70);
\draw (0,70) -- (0,80);
\draw (0,80) -- (5,89);
\draw (5,89) -- (14,94);
\draw (5,89) -- (5,89);
\draw (5,89) -- (10,98);
\draw (0,80) -- (0,80);
\draw (0,80) -- (0,90);
\draw (0,90) -- (-5,99);
\draw (0,90) -- (0,90);
\draw (0,90) -- (0,100);
\draw (0,100) -- (-5,109);
\draw (-5,109) -- (-5,119);
\draw (-5,109) -- (-5,109);
\draw (-5,109) -- (-10,118);
\draw (0,100) -- (0,100);
\draw (0,100) -- (0,110);
\draw (0,110) -- (5,119);
\draw (0,110) -- (0,110);
\draw (0,110) -- (0,120);
\draw (0,120) -- (-5,129);
\draw (0,120) -- (0,120);
\draw (0,120) -- (0,130);
\draw (0,130) -- (5,139);
\draw (5,139) -- (5,149);
\draw (5,139) -- (5,139);
\draw (5,139) -- (10,148);
\draw (10,148) -- (19,153);
\draw (19,153) -- (24,162);
\draw (19,153) -- (19,153);
\draw (19,153) -- (28,158);
\draw (10,148) -- (10,148);
\draw (10,148) -- (15,157);
\draw (15,157) -- (24,162);
\draw (15,157) -- (15,157);
\draw (15,157) -- (20,166);
\draw (20,166) -- (20,176);
\draw (20,166) -- (20,166);
\draw (20,166) -- (25,175);
\draw (25,175) -- (34,180);
\draw (34,180) -- (44,180);
\draw (34,180) -- (34,180);
\draw (34,180) -- (43,185);
\draw (43,185) -- (48,194);
\draw (43,185) -- (43,185);
\draw (43,185) -- (52,190);
\draw (52,190) -- (57,199);
\draw (57,199) -- (66,204);
\draw (57,199) -- (57,199);
\draw (57,199) -- (62,208);
\draw (62,208) -- (62,218);
\draw (62,208) -- (62,208);
\draw (62,208) -- (67,217);
\draw (52,190) -- (52,190);
\draw (52,190) -- (61,195);
\draw (61,195) -- (66,204);
\draw (61,195) -- (61,195);
\draw (61,195) -- (70,200);
\draw (25,175) -- (25,175);
\draw (25,175) -- (30,184);
\draw (30,184) -- (30,194);
\draw (30,184) -- (30,184);
\draw (30,184) -- (35,193);
\draw (35,193) -- (35,203);
\draw (35,203) -- (40,212);
\draw (35,203) -- (35,203);
\draw (35,203) -- (35,213);
\draw (35,193) -- (35,193);
\draw (35,193) -- (40,202);
\draw (40,202) -- (40,212);
\draw (40,202) -- (40,202);
\draw (40,202) -- (45,211);
\draw (0,130) -- (0,130);
\draw (0,130) -- (0,140);
\draw (0,140) -- (5,149);
\draw (0,140) -- (0,140);
\draw (0,140) -- (0,150);
\draw (0,150) -- (-5,159);
\draw (-5,159) -- (-5,169);
\draw (-5,159) -- (-5,159);
\draw (-5,159) -- (-10,168);
\draw (-10,168) -- (-19,173);
\draw (-10,168) -- (-10,168);
\draw (-10,168) -- (-15,177);
\draw (0,150) -- (0,150);
\draw (0,150) -- (0,160);
\draw (0,160) -- (-5,169);
\draw (0,160) -- (0,160);
\draw (0,160) -- (0,170);
\draw (0,170) -- (5,179);
\draw (5,179) -- (5,189);
\draw (5,179) -- (5,179);
\draw (5,179) -- (10,188);
\draw (10,188) -- (19,193);
\draw (19,193) -- (29,193);
\draw (19,193) -- (19,193);
\draw (19,193) -- (28,198);
\draw (28,198) -- (33,207);
\draw (28,198) -- (28,198);
\draw (28,198) -- (37,203);
\draw (10,188) -- (10,188);
\draw (10,188) -- (15,197);
\draw (15,197) -- (24,202);
\draw (15,197) -- (15,197);
\draw (15,197) -- (20,206);
\draw (20,206) -- (20,216);
\draw (20,206) -- (20,206);
\draw (20,206) -- (25,215);
\draw (25,215) -- (25,225);
\draw (25,225) -- (20,234);
\draw (25,225) -- (25,225);
\draw (25,225) -- (25,235);
\draw (25,215) -- (25,215);
\draw (25,215) -- (30,224);
\draw (30,224) -- (39,229);
\draw (30,224) -- (30,224);
\draw (30,224) -- (35,233);
\draw (0,170) -- (0,170);
\draw (0,170) -- (0,180);
\draw (0,180) -- (5,189);
\draw (0,180) -- (0,180);
\draw (0,180) -- (0,190);
\draw (0,190) -- (5,199);
\draw (5,199) -- (14,204);
\draw (5,199) -- (5,199);
\draw (5,199) -- (10,208);
\draw (10,208) -- (10,218);
\draw (10,208) -- (10,208);
\draw (10,208) -- (15,217);
\draw (0,190) -- (0,190);
\draw (0,190) -- (0,200);
\draw (0,200) -- (5,209);
\draw (0,200) -- (0,200);
\draw (0,200) -- (0,210);
\draw (0,210) -- (-5,219);
\draw (-5,219) -- (-5,229);
\draw (-5,219) -- (-5,219);
\draw (-5,219) -- (-10,228);
\draw (-10,228) -- (-19,233);
\draw (-10,228) -- (-10,228);
\draw (-10,228) -- (-15,237);
\draw (-15,237) -- (-15,247);
\draw (-15,247) -- (-10,256);
\draw (-15,247) -- (-15,247);
\draw (-15,247) -- (-15,257);
\draw (-15,237) -- (-15,237);
\draw (-15,237) -- (-20,246);
\draw (-20,246) -- (-20,256);
\draw (-20,246) -- (-20,246);
\draw (-20,246) -- (-25,255);
\draw (-25,255) -- (-34,260);
\draw (-34,260) -- (-44,260);
\draw (-34,260) -- (-34,260);
\draw (-34,260) -- (-43,265);
\draw (-25,255) -- (-25,255);
\draw (-25,255) -- (-30,264);
\draw (-30,264) -- (-39,269);
\draw (-30,264) -- (-30,264);
\draw (-30,264) -- (-35,273);
\draw (-35,273) -- (-44,278);
\draw (-44,278) -- (-49,287);
\draw (-44,278) -- (-44,278);
\draw (-44,278) -- (-53,283);
\draw (-53,283) -- (-63,283);
\draw (-53,283) -- (-53,283);
\draw (-53,283) -- (-62,288);
\draw (-62,288) -- (-67,297);
\draw (-67,297) -- (-76,302);
\draw (-67,297) -- (-67,297);
\draw (-67,297) -- (-72,306);
\draw (-62,288) -- (-62,288);
\draw (-62,288) -- (-71,293);
\draw (-71,293) -- (-76,302);
\draw (-71,293) -- (-71,293);
\draw (-71,293) -- (-80,298);
\draw (-80,298) -- (-90,298);
\draw (-80,298) -- (-80,298);
\draw (-80,298) -- (-89,303);
\draw (-35,273) -- (-35,273);
\draw (-35,273) -- (-40,282);
\draw (-40,282) -- (-40,292);
\draw (-40,282) -- (-40,282);
\draw (-40,282) -- (-45,291);
\draw (-45,291) -- (-45,301);
\draw (-45,301) -- (-40,310);
\draw (-45,301) -- (-45,301);
\draw (-45,301) -- (-45,311);
\draw (-45,311) -- (-50,320);
\draw (-45,311) -- (-45,311);
\draw (-45,311) -- (-45,321);
\draw (-45,291) -- (-45,291);
\draw (-45,291) -- (-50,300);
\draw (-50,300) -- (-59,305);
\draw (-50,300) -- (-50,300);
\draw (-50,300) -- (-55,309);
\draw (-55,309) -- (-64,314);
\draw (-64,314) -- (-69,323);
\draw (-64,314) -- (-64,314);
\draw (-64,314) -- (-73,319);
\draw (-55,309) -- (-55,309);
\draw (-55,309) -- (-60,318);
\draw (-60,318) -- (-60,328);
\draw (-60,318) -- (-60,318);
\draw (-60,318) -- (-65,327);
\draw (0,210) -- (0,210);
\draw (0,210) -- (0,220);
\draw (0,220) -- (5,229);
\draw (0,220) -- (0,220);
\draw (0,220) -- (0,230);
\draw (0,230) -- (-5,239);
\draw (-5,239) -- (-5,249);
\draw (-5,239) -- (-5,239);
\draw (-5,239) -- (-10,248);
\draw (-10,248) -- (-19,253);
\draw (-10,248) -- (-10,248);
\draw (-10,248) -- (-15,257);
\draw (0,230) -- (0,230);
\draw (0,230) -- (0,240);
\draw (0,240) -- (5,249);
\draw (0,240) -- (0,240);
\draw (0,240) -- (0,250);
\draw (0,250) -- (-5,259);
\draw (0,250) -- (0,250);
\draw (0,250) -- (0,260);
\draw (0,260) -- (-5,269);
\draw (-5,269) -- (-5,279);
\draw (-5,269) -- (-5,269);
\draw (-5,269) -- (-10,278);
\draw (-10,278) -- (-10,288);
\draw (-10,288) -- (-5,297);
\draw (-10,288) -- (-10,288);
\draw (-10,288) -- (-10,298);
\draw (-10,278) -- (-10,278);
\draw (-10,278) -- (-15,287);
\draw (-15,287) -- (-15,297);
\draw (-15,287) -- (-15,287);
\draw (-15,287) -- (-20,296);
\draw (-20,296) -- (-29,301);
\draw (-29,301) -- (-39,301);
\draw (-29,301) -- (-29,301);
\draw (-29,301) -- (-38,306);
\draw (-20,296) -- (-20,296);
\draw (-20,296) -- (-25,305);
\draw (-25,305) -- (-25,315);
\draw (-25,305) -- (-25,305);
\draw (-25,305) -- (-30,314);
\draw (-30,314) -- (-39,319);
\draw (-30,314) -- (-30,314);
\draw (-30,314) -- (-35,323);
\draw (0,260) -- (0,260);
\draw (0,260) -- (0,270);
\draw (0,270) -- (-5,279);
\draw (0,270) -- (0,270);
\draw (0,270) -- (0,280);
\draw (0,280) -- (5,289);
\draw (5,289) -- (14,294);
\draw (5,289) -- (5,289);
\draw (5,289) -- (10,298);
\draw (0,280) -- (0,280);
\draw (0,280) -- (0,290);
\draw (0,290) -- (5,299);
\draw (0,290) -- (0,290);
\draw (0,290) -- (0,300);
\end{tikzpicture}}

    \end{subfigure}
    \caption{Resultate Stochastisches L-System}\label{fig:Stochastic}
\end{figure}


\subsubsection{Erweiterung in 3D}
Die L-Systeme aus den vorherigen Abschnitten haben stets zweidimensionale Strukturen erstellt.
Dabei wurden die Symbole \texttt{+} und \texttt{-} genutzt um die aktuelle Richtung des Zeichenkopfes in der Ebene zu rotieren.
Für Strukturen in drei Dimensionen kann diese Idee weiterverwendet werden.
Hierbei werden jeweils zwei Symbole für Rotationen um jede der drei Achsen eingeführt.
Die Symbole \texttt{+} und \texttt{-} rotieren den Zeichenkopf um die $z$-Achse, \texttt{\&} und \texttt{\textasciicircum} rotieren um die $x$-Achse und \texttt{/} und \texttt{\textbackslash} rotieren um die $y$-Achse.
Eine Visualisierung dieser Idee ist in Abbildung~\ref{fig:L-System 3D Rotation} zu sehen.
% Add 3D rotation tikz
\begin{figure}[ht]
    \centering
    % \resizebox{\linewidth-3cm}{!}{%
        
    % }
    \includegraphics[width=0.5\linewidth]{example-image-a}
    \caption{L-System 3D Rotation}\label{fig:L-System 3D Rotation}
\end{figure}

Ein Beispiel für ein L-System, das einen dreidimensionalen Baum erzeugt ist im Folgenden gegeben:
\begin{itemize}
    \item Nicht-Terminalsymbole $N=\{F\}$
    \item Terminalsymbole $T=\{\texttt{+},\texttt{-},\texttt{\&},\texttt{\textasciicircum},\texttt{/},\texttt{\textbackslash},\texttt{[},\texttt{]}\}$
    \item Startstring $S=FFFA$
    \item Produktionen:
    \begin{align*}
        A\rightarrow &~\texttt{[}B\texttt{]////[}B\texttt{]////}B \\
        B\rightarrow &~\texttt{\&}FFFA
    \end{align*}
\end{itemize}

Mit diesem L-System kann ein dreidimensionaler Baum wie in Abbildung~\ref{fig:L-System 3D Unity} in Unity erzeugt werden.
Es wurden 7 Ableitungsschritte durchgeführt und es wurde jeweils um 28 Grad um die Achsen rotiert.
% TODO add picture and L-System from Unity when its fixed
\begin{figure}[ht]
    \centering
    % \resizebox{\linewidth-3cm}{!}{%
        
    % }
    \includegraphics[width=0.5\linewidth]{chapters/02_Grundlagen/L_System/L_System_3D_Tree.png}
    \caption{L-System 3D Baum}\label{fig:L-System 3D Unity}
\end{figure}


\subsection{Auswertung}
Die Auswertung eines L-Systems wird in diesem Abschnitt beschrieben.


\begin{algorithm}[H]
    \begin{algorithmic}[1]
        \footnotesize
        \REQUIRE{Grammatik $(N,T,S,P)$, Iterationen $i$}
        \ENSURE{Wort $w$ nach $i$ Iterationen}
        \STATE{$w \gets S$}
        \FOR{$i$ Iterationen}
        \STATE{$temp\gets \emptyset$}
        \FOR{$c\in w$}
        \IF{$c\in N$}
        \FOR{$(p,\Omega,R)\in P$ mit $c$ == $p$}
        \STATE{Wähle eine Ersetzung $r\in R$ mit Wahrscheinlichkeit $\omega\in \Omega$}
        \STATE{Hänge $r$ an $temp$ an}
        \ENDFOR
        \ELSE
        \STATE{Hänge $c$ an $temp$ an}
        \ENDIF
        \ENDFOR
        \STATE{$w\gets temp$}
        \ENDFOR
        \RETURN{$w$}
    \end{algorithmic}
    \caption{L-System Auswertung}\label{alg:L-System Auswertung}
\end{algorithm}




\section{Metaball} \label{sec:metaball}
Ein Ansatz zur Erzeugung von visuell ansprechender, natürlich wirkender Geometrie ist der Metaball \cite{metaballArticle}.
Die grundlegende Idee dahinter ist es mehrere Kugeln im Raum zu platzieren und diese ineinander "verschmelzen" zu lassen, sodass sich daraus ein größeres Objekt ergibt (siehe \ref{fig:2dMetaball}).

\begin{figure}[ht]
    \centering
    \includegraphics[width=0.5\linewidth]{chapters/02_Grundlagen/Metaball/metaball_2d}
    \caption{2D Beispiel eines Metaballs bestehend aus drei Kugeln}\label{fig:2dMetaballl}
\end{figure}

Der Metaball $M$, bestehend aus $n$ Kugeln, ist durch eine Funktion gegeben, die jedem Punkt $\vec{x}$ im Raum einen Wert zuordnet.
\[M(\vec{x})=\sum_{i=0}^{n}f_i(\vec{x})\]
Diese setzt sich aus den Funktionen jeder der $n$ Kugeln zusammen. Für eine einzelne Kugel lässt sich dieser Wert als Abstand zu deren Zentrum interpretieren. \\
Zusätzlich benötigt man einen Schwellenwert (typischerweise ist dieser 1), der beschreibt, welcher Wert für $M$ den Übergang zwischen innerhalb und außerhalb des Metaballs festlegt.
Ist $M<Schwellenwert$ liegt der Punkt außerhalb des Metaballs, ist $M>Schwellenwert$ liegt er innerhalb. $M=Schwellenwert$ beschreibt somit die Oberfläche des Körpers.\\

Eine mögliche Metaball-Funktion $f_i$ ist
\[f_i(\vec{x}) = \frac{1}{||\vec{c_i}-\vec{x}||} = \frac{1}{r_i(\vec{x})}\]
Wobei $\vec{c_i}$ das Zentrum von Kugel $i$ beschreibt und $r_i$ den Abstand zwischen $\vec{x}$ und $\vec{c_i}$.

Um aus dieser Funktion ein diskretes Mesh zu erzeugen, lässt sich der Raum mit Hilfe des Marching-Cubes-Algorithmus Abtasten um die Oberfläche zu bestimmen.
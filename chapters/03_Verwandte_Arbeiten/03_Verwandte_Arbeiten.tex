\chapter{Verwandte Arbeiten}
\label{Verwandte_Arbeiten}

Beschreiben warum andere Quellen nicht ausreichend waren und weshalb der eigene Ansatz jene fehlende Themen ergänzt.

\section{Creature Generation using Genetic Algorithms and Auto-Rigging\cite{Hudson2013CreatureGU}}
In seiner Thesis beschreibt Hudson einen möglichen Ansatz zur prozeduralen Generierung von Kreaturen, basierend auf Parametern in Form von Minima und Maxima für Dimensionen einzelner Körperteile.\\ Zunächst wird ein Torso generiert, der aus drei mit Hilfe der Parameter zufällig skalierten Segmenten besteht. Daran werden weitere Körperteile wie Arme, Beine, Kopf und Flügel jeweils an der richtigen Stelle platziert. Die Anzahl und Ausmaße dieser werden ebenfalls zufällig bestimmt.\\
Vorausgesetzt werden hierbei Annahmen über die prinzipelle Struktur des Skeletts. Hudson beschränkt sich hierbei auf humanoide Zweibeiner und ein allgemeines Vierbeiner Modell. Welche Positionen und Rotationen dabei genau verwendet werden wird nicht spezifiziert. Dieser Aspekt lässt weiter Raum für Optimierung offen und wirft die Frage danach auf, inwiefern sich auch diese Werte randomisieren lassen. Des weiteren spricht Hudson selbst die mögliche Erweiterung auf andere Typen von Kreaturen an.\\
Außerdem können die daraus erzeugten Kreaturen mit Hilfe eines genetischen Algorithmus kombiniert und mutiert werden, was für mehr Variation sorgen soll, während gleichzeitig gute Ergebnisse weiterentwickelt werden können.\\
Die Motivation besteht darin, einem Experten Vorlagen zur manuellen Modellierung einer glaubwürdigen Kreatur zu liefern. Dieser Ansatz müsste erweitert werden um ihn auf unser Ziel anzupassen, wobei die vollständig prozedurale Generierung im Vordergrund steht.

\section{Procedural Generation of 2D Creatures \cite{Janno20182dCG}}
In dieser Thesis wird unter anderem ein möglicher Ansatz zur Generierung eines Meshes um das zuvor erzeugte Skelett einer Kreatur. Dies geschieht mit Hilfe von Metaballs.\\
Metaballs bilden die implizite Definition einer Oberfläche, die jedem Punkt im Raum einen Wert $F(x,y,z)$ zuweist. Überschreitet dieser Wert einen festgelegten Schwellenwert(häufig 1), liegt der Punkt außerhalb des Meshes, liegt er darunter, liegt der Punkt innerhalb des Meshes. Die Oberfläche liegt dort wo $F$ gleich dem Schwellenwert ist. $F$ sezt sich aus den Gleichungen mehrerer Kugeln $i\in \{ 1, ..., N\}$ zusammen, sodass diese ineinander verschmelzen und eine Glatte Fläche bilden, wenn sie einander nahegelegen sind.
\[F(x,y,z) = \sum_{i=1}^{N}{f_i(x,y,z)}\]
Es können beliebige Distanzfunktionen für die einzelnen Kugeln verwendet werden, Janno hat sich auf die folgende festgelegt.
\[f_i(x,y,z)=exp(\frac{B_ir_i^2}{R_i^2}-B_i)\]
mit $B_i=-0.5$. Dabei ist $r_i$ die euklidische Distanz des Punktes zum Mittelpunkt der Kugel, und $R_i$ der Radius der Kugel.\\
Die Idee zur Platzierung der Kugeln entlang der Kreatur ist, sie entlang jedes Segmentes mit einem Abstand zu positionieren, der garantiert, dass sie eine zusammenhängende Fläche bilden. Nimmt man einen gleichen Radius $R$ von zwei Kugeln an, lässt sich der maximale Abstand dieser zu einem Punkt bestimmen, der genau den Schwellenwert(=1) erreicht.
\[2\cdot f(x,y,z) = 1\]
Durch Umformen erhält man einen Wert für den Abstand zum Mittelpunkt, der verdoppelt den maximalen Abstand der beiden Kugeln ergibt. Im Fall der verwendeten Funktion bedeutet dies einen Abstand von $R \cdot 3.13$.\\
Die Kugeln werden anschließend mit gleichem Abstand entlang des Segmentes platziert, sodass dieser den maximalen Abstand nicht überschreitet.

% Niklas
\section{Introduction}
\subsection{ml-Agents}
Das Unity Machine Learning Agents Toolkit, kurz ml-Agents, ist ein von Unity entwickeltes und unter der Apache License 2.0 lizensiertes open-source Projekt, dass Implementierungen von beliebten Reinforment Learning Algorithmen zur Verfügung stellt.
Das Toolkit wurde am 19. September 2017 vorgestellt und wurde seitdem stetig weiterentwickelt. Die aktuellste Version ist Release 19, welcher am 14. Januar 2022 veröffentlicht wurde. Diese Version bietet neben Implementierungen für die Deep Reinforcement Learning Algorithmen PPO, SAC und MA-POCA auch Implementierungen für die Imitation Learning Algorithmen BC und GAIL.\\

\noindent Neben den Implementierungen der Algorithmen stellt das Toolkit auch eine Python API zur Verfügung. Mit dieser API können eigene Agenten mit den zur Verfügung gestellten Algorithmen trainiert werden. Dabei unterstützt die API sowohl Diskrete als auch Kontinuierliche Aktions- und Beobachtungsräumne. Es unterstützt außerdem das Platzieren von mehreren Agenten in einer Umgebung. Diese können sowohl das selbe Verhalten lernen, um das Training zu beschleunigen, oder verschiedene Verhalten, zum Beispiel zum Trainieren der Charaktere in asymmetrischen Spielen. \\

\noindent Neben der nach der Python API erstellten Umgebung benötigt ml-Agents zum Starten des Trainingsvorgangs noch eine Konfigurations Datei. Diese wird für ml-Agents als eine yaml Datei zur Verfügung gestellt. Diese Datei beinhaltet eine Liste von Verhalten und jeder Agent in der Umgebung muss einem der Verhalten entsprechen. Dies ermöglicht es für verschiedene Verhalten verschiedene Konfigurationen für den Ablauf des Trainings und den Aufbau des Netzwerks anzugeben. 
Die Konfiguration definiert den zu verwendenden Algorithmus, die Anzahl der Schritte, die während des Trainings in der Umgebung ausgeführt werden sollen, und wie oft Checkpoints des Netzes gespeichert werden sollen. Zusätzlich beinhaltet es verschiedene Abschnitte, die Teilaspekte des Trainings beeinflussen. \\
Der Hyperparameter Abschnitt der Datei beschreibt die Parameter des Trainings, wie die Batchgröße, die Buffergröße und die globalen Parameter des ausgewählten Algorithmus, wie zum Beispiel die Lernrate oder das beta und epsilon, bei PPO, oder das tau, bei SAC. \\
Der Network Settings Abschnitt beschreibt die Größe und Anzahl der Schichten in dem Netzwerk. Zusätzlich kann hier angegeben werden, ob die Werte normalisiert werden sollen.\\

\noindent Sowohl die Checkpoints als auch das finale Netzwerk werden als onnx Datei gespeichert, die dann, zum Beispiel mit Unity Barracuda, geladen werden kann, um das Netzwerk zu verwenden.\\
Zum Analysieren und Bewerten des Trainings erhebt ml-Agents während des Trainings verschiedene Daten, wie den durchschnittlichen Reward und die Loss Werte der verschiedenen Netzwerke. Diese werden in einem Tensorboard gespeichert, welches in dem selben Ordner wie die onnx Datei gefunden werden kann. 

\subsection{neroRL}

NeroRL ist ein unter der MIT Lizenz lizensiertes Reinforment Learning Framework, dass seinen Fokus auf verschiedene Varianten des Proximal Policy Optimization (PPO) Algorithmus legt. Es basiert auf dem Code von dem ml-Agents Toolkit und verwendet dessen Python API, um mit den Umgebungen zu kommunizieren. \\
NeroRL stellt neben der regulären Variante von PPO auch eine Implementierung mit Rekkurenz zur Verfügung. Zusätzlich kann bei allen Implementierungen der Agent entweder mit geteilten oder mit getrennten Netzwerken und Gradienten für den Aktor und den Kritiker trainiert werden. \\
Diese Implementierungen können verwendet werden, um in der bereits existierenden ObstacleTower Umgebung oder in beliebigen openAIGym Umgebungen zu trainieren. Durch das aufbauen auf dem ml-Agents Toolkit können zusätzlich auch Agenten in Umgebungen trainiert werden, die mit der Python API von ml-Agents erstellt wurden.\\
Allerdings stellt NeroRL noch einige zusätzliche Anforderungen an diese Umgebungen. So dürfen die Observationsräume nur Vektor und Visuelle Beobachtungen enthalten. Außerdem werden nur Diskrete und Multidiskrete Aktionsräume unterstützt. NeroRL kann also nicht verwendet werden, um Agenten in einer Umgebung mit kontinuierlichen Aktionsräumen zu trainieren.
Des weiteren unterstützt neroRL nur genau einen Agenten pro Umgebung. Es ist also nicht möglich das Training zu beschleunigen, indem mehrere Agenten in einer Umgebung das selbe Verhalten lernen. Zum Beschleunigen des Trainings kann bei neroRL stattdessen über ein Attribut in der Konfigurationsdatei angegeben werden, dass mit mehreren Instanzen der Umgebung gleichzeitig trainiert werden soll. So würden bei neroRL zum Beispiel anstelle von einer Umgebung mit zehn Agenten zehn Instanzen der Umgebung mit je einem Agenten ausgeführt.\\

\noindent Zum Starten des Trainingsvorgangs muss dem Framework eine Konfigurationsdatei im yaml Format übergeben werden, welche alle relevanten Informationen für das Training enthält.\\
Der erste Abschnitt enthält Informationen über die Umgebung. Wenn für das Training in Build verwendet werden soll, dann wird der Pfad dazu hier angegeben. Außerdem können in diesem Abschnitt die Reset Parameter angegeben werden.\\
Der zweite Abschnitt enthält die Informationen über das Netzwerk. Dazu gehören neben der Anzahl der Ebenen und deren Größe
auch die Aktivierungs und Kodierungsfunktionien. Der Pfad unter dem das Model gespeichert werden soll und der Abstand der Checkpoints wird ebenfalls in diesem Abschnitt festgelegt.\\
Der dritte Abschnitt beinhaltet Informationen zur Evaluierung eines Modells. Dazu gehört mit wie vielen Instanzen evaluiert werden soll und mit welchen Seeds.\\
Im vierten Abschnitt wird festgelegt wie viele parallele Instanzen der Umgebung zum trainieren verwendet werden sollen und wie viele Schritte jede Instanz pro Update machen soll. Das Produkt dieser beiden Werte ergibt die Batchgröße für das Training.\\
Der letzte Abschnitt enthält die Informationen zum eigentlichen Training. In diesem Abschnitt werden der zu verwendende Algorithmus und die Werte für die Hyperparameter von diesem festgelegt. Die Anzahl an durchzuführenden Updates und die Anzahl an Epochs pro Update werden ebenfalls in diesem Abschnitt festgelegt.\\

\noindent Zum Analysieren und Bewerten erhebt neroRL während des Trainings verschiedene Kenndaten, wie den durchschnittlichen Reward und die Loss Werte der verschiedenen Netzwerke. Diese werden in einem Tensorboard gespeichert, welches standardmäßig in einem summaries Ordner zu finden ist.


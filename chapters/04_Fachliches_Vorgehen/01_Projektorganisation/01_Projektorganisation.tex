\section{Projektorganisation}

Wie sind wir vorgegangen... auf alles bezogen. Wer hat an welchen Kapiteln mitgearbeitet.

%Thomas
? Während der Projektgruppe haben sich ... Untergruppen/Teams für die jeweiligen Bereiche ... herausgestellt.

Für die Projektorganisation haben sich ein Projektleiter und jeweils 2 Projektmanager aus den Gruppen der \textbf{Creature-Generator} und \textbf{Creature-Animator} einberufen. Wöchentlich findet ein Jour-Fixe statt um gemeinsam über die aktuellen Entwicklungen zu sprechen und Herausforderungen gemeinsam anzugehen und zu lösen. Der Projektleiter sollte darin eine Moderator- und Organisationsrolle einnehmen, die Gruppen und Themen zusammen- und im Überblick behalten und somit potentielle Konflikte frühzeitig erkennen und auflösen. Die Projektmanager übernehmen eine analoge Aufgabe im Rahmen ihrer jeweiligen Gruppen und bilden somit gemeinsam mit dem Projektleiter eine Struktur in der die Übersicht über aktuelle Themen einfach beibehalten werden kann. Im Anschluss an das Jour-Fixe treffen sich die vier Projektmanager und der Projektleiter, bereiten für den nächsten Termin des Jour-Fixe die zu besprechenden Themen vor und klären gegebenenfalls organisatorische Einzelheiten um somit die restliche Gruppe der Teilnehmer zu entlasten. Während der Jour-Fixe wird von jedem der Teilnehmer abwechselnd eine Dokumentation des Treffens manifestiert um somit Ergebnisse aber auch ToDo's festhalten und einsehen zu können. Für das wöchentliche Treffen wurde ein Tagesprotokoll (Fig. ) genutzt, welches von dem Projektleiter durch Bildschirmübertragung an die Teilnehmer präsentiert und durchgesprochen wurde. Das Tagesprotokoll konnte jederzeit durch jeden Teilnehmer modifiziert und ergänzt werden, falls zu besprechende Themen durch den jeweiligen Teilnehmer aufgekommen sind.

\paragraph{Organisationswerkzeuge}
Für die weitere Organisation (der Entwicklung und Implementierung) der Projektgruppe wurden verschiedene Tools und Hilfsmittel eingesetzt, die im Folgenden beschrieben werden.

\begin{itemize}
	\item \textbf{Discord: } Discord wurde als Basis-Kommunikationsmittel genutzt. Hier wurden für die entsprechenden Phasen und Themenbereiche, sowie Gruppen eigene Text- und Sprach-Kanäle erstellt (Fig. (?)). Hier wurden auch aktuelle (organisatorische-) Themen aufgegriffen und besprochen. Der wöchentliche Jour-Fixe Termin wurde ebenfalls in Discord abgehalten.
	\item \textbf{GitHub: } GitHub wurde als Entwicklungs-, Repository- und Speicherplattform genutzt. Die Dokumentationen des wöchentlichen Jour-Fixe sind im GitHub-Wiki abgelegt und zusätzlich in einem Discord Textkanal gespeichert. omit wurde implizit auch durch die Verfügbarkeit auf zwei Plattformen ein Backup der Dokumentationen realisiert, falls im worst-case eine der beiden Plattformen ausfallen sollte. Für die Projektgruppe wurde eine GitHub-Organisation gegründet (\href{https://github.com/PG649-3D-RPG}{Link zur Organisation}). Pro Gruppe bzw. Themenbereich wurde zunächst ein Repository erstellt. Nach Fertigstellung einzelner Versionen werden diese als Pakete exportiert und in einem Main-Repository (\href{}{Link}) zum fertigen Spiel zusammengestellt.
	\item \textbf{Google-Kalender: } Um insbesondere in der vorlesungsfreien Zeit eine reibungslose Organisation und Ressourcenallozierung gewährleisten zu können, wurde Google-Kalender genutzt, in dem sich alle Teilnehmenden in einem Kalender gesammelt haben und ihre jeweiligen Universitäts- und Urlaubsbezogenen Termine eingetragen haben.
	\item \textbf{Status-Quo-Übersicht: } In der Anfangszeit der Projektgruppe hat sich herausgestellt, dass nur schwierig der Überblick über den Gesamtfortschritt der Projektgruppe zu halten ist. Um den Fortschritt insbesondere auch außerhalb der Projektorganisation zu festzuhalten, wurde ein Status-Quo Dokument entwickelt. Dieses stellt bereits fertiggestellte, momentan laufende und potentiell in Zukunft zu behandelnde Projekte und Themengebiete übersichtlich dar. Dafür wurde mit Hilfe des Tools Graphviz ein simpler Graphen erzeugt der zudem am Anfang jedes Monats durch die Management-Runde aktuell gehalten wurde. Abbildung \ref{fig:status-quo} zeigt den Status-Quo zum Zeitpunkt der Erstellung des Abschlussberichtes.
	\begin{figure}
		\centering
		\includegraphics[width=0.7\linewidth]{example-image-a}
		\caption{Stand des Status-Quo am (Datum)}
		\label{fig:status-quo}
	\end{figure}
	\item \textbf{Projekt-Timeline: } Zu Beginn der Projektgruppe wurde eine Projekt-Timeline erstellt, welche für das jeweilige Semester die kurz- und langfristigen Aufgaben der entsprechenden Gruppen in tabellarischer Form festhalten sollte. Da dieses Hilfsmittel jedoch in dem ersten Semester keine Verwendung fand, wurde es zu Anfang der zweiten Hälfte der Projektgruppe als obsolet eingestuft und verworfen. Die kurz- und langfristigen Aufgaben wurden stattdessen über das Status-Quo Dokument und die wöchentlichen Jour-Fixe Dokumentationen festgehalten.
\end{itemize}



\subsection{Creature Animator}
% Carsten
Die Gruppe der Creature Animator hat sich in zwei Untergruppen aufgeteilt. In der ersten Phase hat sich die erste Untegruppe damit beschäftigt, den ML-Agents Walker in eine neue Trainingsumgebung einzubauen und die Skripte dynamischer zu gestalten, damit diese in der zweiten Arbeitsphase verwendet und erweitert werden konnten. Währenddessen versuchte die andere Untergruppe den ML-Agents Walker das Schlagen beizubringen. Die beiden Untergruppen haben sich wöchentlich mittwochs getroffen, um von ihren Fortschritten und Problemen zu berichten. Dabei wurden die Ergebnisse in Protokollen festgehalten, welche in einem GitHub Wiki abgelegt wurden.

In der zweiten Phase, welche nach der Bereitstellung der ersten generierten Kreaturen von der Creature Generator Gruppe begann, veränderten sich die Aufgabenbereiche der beiden Untergruppen. Die \enquote{Schlagen}-Gruppe arbeitete seit dem an einer Erweiterung von Nero-RL, sodass Nero-RL anstelle von ML-Agents zum Trainieren der Kreaturen genutzt werden kann. Die Aufgabe der \enquote{Trainingsumgebung}-Gruppe war es den neuen Kreaturen das Fortbewegen beizubringen und der Creature Generator Gruppe Feedback zu den Kreaturen zu geben. Dabei arbeiteten die Gruppenmitglieder an verschiedenen kleineren Aufgaben. Jan beschäftigte sich mit dem Training und dem Finden und Ausprobieren neuer Rewardfunktionen, Nils arbeitete an der dynamischen Generierung von Arenen und dem Landen von Konfigurationeinstellungen aus Dateien und Carsten testete verschiedene Parameter aus und implementierte das Erstellen von NavMeshes zur Laufzeit. In der zweiten Phase lösten \enquote{On-Demand}-Treffen die regelmäßigen Treffen zwischen den beiden Untergruppen ab, um mehr Zeit zum Arbeiten an den Aufgaben zu haben. Zudem wurden anstelle der Treffen nur noch die wichtigsten Punkte protokolliert. Ansonsten wurden Probleme und Fehler direkt als Issue in den entsprechenden GitHub Repositories hinterlegt.  
\begin{table}[]
	\centering
	\begin{tabular}{l|l}
		\begin{tabular}[c]{@{}l@{}}\enquote{Trainingsumgebung/Movement}-Gruppe\end{tabular} & \enquote{Schlagen/Nero-RL}-Gruppe \\ \hline
		Carsten Kellner                                                       & Jannik Stadtler  \\
		Jan Beier                                                             & Niklas Haldorn   \\
		Nils Dunker                                                                 &                 
		\end{tabular}
		\caption{Die zwei Untergruppen und ihre Mitglieder}
	\end{table}



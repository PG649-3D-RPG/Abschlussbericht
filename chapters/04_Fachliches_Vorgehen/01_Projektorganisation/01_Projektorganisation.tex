\section{Projektorganisation}

Wie sind wir vorgegangen... auf alles bezogen. Wer hat an welchen Kapiteln mitgearbeitet.

\subsection{Creature Animator}
% Carsten
Die Gruppe der Creature Animator hat sich in zwei Untergruppen aufgeteilt. In der ersten Phase hat sich die erste Untegruppe damit beschäftigt, den ML-Agents Walker in eine neue Trainingsumgebung einzubauen und die Skripte dynamischer zu gestalten, damit diese in der zweiten Arbeitsphase verwendet und erweitert werden konnten. Währenddessen versuchte die andere Untergruppe den ML-Agents Walker das Schlagen beizubringen. Die beiden Untergruppen haben sich wöchentlich mittwochs getroffen, um von ihren Fortschritten und Problemen zu berichten. Dabei wurden die Ergebnisse in Protokollen festgehalten, welche in einem GitHub Wiki abgelegt wurden.

In der zweiten Phase, welche nach der Bereitstellung der ersten generierten Kreaturen von der Creature Generator Gruppe begann, veränderten sich die Aufgabenbereiche der beiden Untergruppen. Die \enquote{Schlagen}-Gruppe arbeitete seit dem an einer Erweiterung von Nero-RL, sodass Nero-RL anstelle von ML-Agents zum Trainieren der Kreaturen genutzt werden kann. Die Aufgabe der \enquote{Trainingsumgebung}-Gruppe war es den neuen Kreaturen das Fortbewegen beizubringen und der Creature Generator Gruppe Feedback zu den Kreaturen zu geben. Dabei arbeiteten die Gruppenmitglieder an verschiedenen kleineren Aufgaben. Jan beschäftigte sich mit dem Training und dem Finden und Ausprobieren neuer Rewardfunktionen, Nils arbeitete an der dynamischen Generierung von Arenen und dem Landen von Konfigurationeinstellungen aus Dateien und Carsten testete verschiedene Parameter aus und implementierte das Erstellen von NavMeshes zur Laufzeit. In der zweiten Phase lösten \enquote{On-Demand}-Treffen die regelmäßigen Treffen zwischen den beiden Untergruppen ab, um mehr Zeit zum Arbeiten an den Aufgaben zu haben. Zudem wurden anstelle der Treffen nur noch die wichtigsten Punkte protokolliert. Ansonsten wurden Probleme und Fehler direkt als Issue in den entsprechenden GitHub Repositories hinterlegt.  
\begin{table}[]
	\centering
	\begin{tabular}{l|l}
		\begin{tabular}[c]{@{}l@{}}\enquote{Trainingsumgebung/Movement}-Gruppe\end{tabular} & \enquote{Schlagen/Nero-RL}-Gruppe \\ \hline
		Carsten Kellner                                                       & Jannik Stadtler  \\
		Jan Beier                                                             & Niklas Haldorn   \\
		Nils Dunker                                                                 &                 
		\end{tabular}
		\caption{Die zwei Untergruppen und ihre Mitglieder}
	\end{table}
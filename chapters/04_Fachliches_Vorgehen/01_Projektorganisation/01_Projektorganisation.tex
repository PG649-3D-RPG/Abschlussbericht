\section{Projektorganisation}

%Thomas
Während der Projektgruppe haben sich ... Untergruppen/Teams für die jeweiligen Bereiche ... herausgestellt.

Für die Projektorganisation haben sich ein Projektleiter und jeweils 2 Projektmanager aus den Gruppen der \textbf{Creature-Generator} und \textbf{Creature-Animator} einberufen. Wöchentlich findet ein Jour-Fixe statt um gemeinsam über die aktuellen Entwicklungen zu sprechen und Herausforderungen gemeinsam anzugehen und zu lösen. Der Projektleiter sollte darin eine Moderator- und Organisationsrolle einnehmen, die Gruppen und Themen zusammen- und im Überblick behalten und somit potentielle Konflikte frühzeitig erkennen und auflösen. Die Projektmanager übernehmen eine analoge Aufgabe im Rahmen ihrer jeweiligen Gruppen und bilden somit gemeinsam mit dem Projektleiter eine Struktur in der die Übersicht über aktuelle Themen einfach beibehalten werden kann. Im Anschluss an das Jour-Fixe treffen sich die vier Projektmanager und der Projektleiter, bereiten für den nächsten Termin des Jour-Fixe die zu besprechenden Themen vor und klären gegebenenfalls organisatorische Einzelheiten um somit die restliche Gruppe der Teilnehmer zu entlasten. Während der Jour-Fixe wird von jedem der Teilnehmer abwechselnd eine Dokumentation des Treffens manifestiert um somit Ergebnisse aber auch ToDo's festhalten und einsehen zu können. Für das wöchentliche Treffen wurde ein Tagesprotokoll (Fig. ) genutzt, welches von dem Projektleiter durch Bildschirmübertragung an die Teilnehmer präsentiert und durchgesprochen wurde. Das Tagesprotokoll konnte jederzeit durch jeden Teilnehmer modifiziert und ergänzt werden, falls zu besprechende Themen durch den jeweiligen Teilnehmer aufgekommen sind.

Für die weitere Organisation (der Entwicklung und Implementierung) der Projektgruppe wurden folgende Tools und Hilfsmittel eingesetzt. 

Discord wurde als Basis-Kommunikationsmittel genutzt. Hier wurden für die entsprechenden Phasen und Themenbereiche, sowie Gruppen eigene Text- und Sprach-Kanäle erstellt (Fig. (?)). Hier wurden auch aktuelle (organisatorische-) Themen aufgegriffen und besprochen. Der wöchentliche Jour-Fixe Termin wurde ebenfalls in Discord abgehalten. Weiterhin wurde GitHub als Entwicklungs-, Repository- und Speicherplattform genutzt. Sowohl in Discord als auch im GitHub-Wiki wurde die wöchentliche Dokumentation des Jour-Fixe abgelegt um somit zwei Alternativen zu haben auf jene Dokumentation zugreifen zu können. Somit wurde implizit auch durch die Verfügbarkeit auf zwei Plattformen ein Backup der Dokumentationen realisiert, falls im worst-case eine der beiden Plattformen ausfallen sollte. Weiterhin wurden in GitHub pro Gruppe bzw. Themenbereich ein Repository erstellt mit der späteren Idee, sobald die einzelnen Gebiete fertiggestellt sind, mit Hilfe von exportierten Paketen in einem Main-Repository die verschiedenen Teile des Spiels zusammenzubringen  \textcolor{red}{$\rightarrow$ in anderes Kapitel auslagern?} Um vor allem in der vorlesungsfreien Zeit eine reibungslose Organisation und Ressourcenallozierung gewährleisten zu können, wurde Google-Kalender genutzt, indem sich alle Teilnehmenden in einem Kalender gesammelt haben und ihre jeweiligen Universitäts- und Urlaubsbezogenen Termine 
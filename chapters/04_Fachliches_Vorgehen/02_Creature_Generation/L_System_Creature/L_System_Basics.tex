Der erste Ansatz zur Generierung von Kreaturen verwendet ein L-System, um die Struktur des Skelettes zu formalisieren.
Dabei wird jede gezeichnete Strecke der turtle als ein Knochen interpretiert.
Hierdurch entsteht stets ein zusammenhängendes Skelett, da sich die turtle nicht fortbewegen kann ohne zu zeichnen.
An Positionen wo die turtle anhält werden Joints gelegt.
Mit diesem Ansatz können nicht nur Skelette von Zweibeinern und Vierbeinern erzeugt werden, sondern auch Kreaturen mit beliebig vielen Armen oder Beinen.

\subsubsection{Generierung der Skelettstruktur}
Um eine Kreatur mithilfe des L-Systems zu generieren wird zunächst die grundlegende Struktur im Startstring definiert.
Hierbei werden ausgehend von der initialen Richtung der turtle, Komponenten des Skeletts wie der Kopf, Arme, der Torso und die Beine definiert.
Für jede dieser Körperteile kann ein eigenes Nicht-Terminalsymbol verwendet werden, um das jeweilige Skelett der Komponente zu definieren.
Mit diesem Ansatz kann jedes Körperteil unabhängig von allen anderen entworfen werden.
Mithilfe eines stochastischen L-Systems können diese zudem zufällige Ausprägungen haben.
Allgemein ist es auch möglich eine zufällige Anzahl an Armen und Beinen zu erzeugen, womit direkt sehr unterschiedliche Kreaturen erzeugt werden können, die aber stets die gleiche Grundstruktur haben.
Der Entwurf der einzelnen Körperteile wird durch die Produktionen des L-Systems definiert.

Der resultierende String des L-Systems wird von der turtle von links nach rechts, den Regeln entsprechend, durchlaufen.
Für jede Fortbewegung der turtle wird jeweils ihre Startposition und die Endposition im dreidimensionalen Raum als ein Tupel gespeichert.
Die Tupel können als Strecken interpretiert werden, die daraufhin als Knochen dargestellt werden.
Am Ende bildet die Liste der Tupel von Start- und Endpunkten die Skelettstruktur.
Um die Joints sinnvoll definieren zu können muss festgehalten werden, welche Produktion ein jeweiliges Tupel erzeugt hat.
Dies ist notwendig, da die Produktionen die Kategorien der Körperteile definieren und die Joints dementsprechend angepasst generiert werden müssen.
Ausgehend von dieser Liste werden die Knochen erstellt und mit den Joints zu einem vollständigen Skelett zusammengesetzt.
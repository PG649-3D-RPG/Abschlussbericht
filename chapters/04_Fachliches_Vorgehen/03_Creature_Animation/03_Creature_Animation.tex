\section{Creature Animation}

\subsection{Trainingsumgebung}
% Carsten
Als Grundlage für die eigene Trainingsumgebung diente die Trainingsumgebung des ML-Agent Walkers. Bei genauerer Betrachtung der Trainingsumgebung des ML-Agent Walkers stellte sich sehr schnell raus, dass diese für unsere Anforderungen zu statisch war, da es zum Beispiel nicht möglich war, die verwendete Kreatur einfach gegen eine andere Kreatur auszutauschen. Zudem mussten neue Arenen aufwendig per Hand erstellt werden und die Kreaturen konnten nur auf einer flachen Ebene trainiert werden. Daher wurde eine eigene dynamischere Trainingsumgebung erstellt, die vor allem die zuvor genannten Punkte umsetzt. Sowohl die Anzahl der Arenen als auch die Kreatur können in der neuen Trainingsumgebung einfach eingestellt werden. Somit können die Arenen vollständig zur Laufzeit generiert werden. Zudem besteht die Möglichkeit neben flachen Terrain auch unebenes Terrain zu verwenden.

Zunächst wurde der ML-Agents Walker zum Testen der neuen Trainingsumgebung verwendet, damit die Kreatur als Fehlerquelle ausgeschlossen werden konnte. Nachdem die Tests mit dem ML-Agents Walker erfolgreich waren, wurde der ML-Agents Walker durch eine neue Kreatur ersetzt, welche mit Hilfe des L-Systems zuvor generiert wurde. Die neue Kreatur wurde ausgiebig in der neuen Trainingsumgebung getestet und mit dem ML-Agents Walker verglichen. Durch das dadurch gewonnene Feedback konnte die Creature Generator Gruppe Anpassungen und Verbesserungen an der Kreatur vornehmen.

Trotz den vielen Änderungen an der Kreatur war es nicht möglich, die Kreatur aus dem L-System zum Laufen zu bringen. Aus diesem Grund ersetzte eine andere Methode, welche von Jona und Markus umgesetzt wurde, das L-System. Mit der neuen Methode wurde auch der Creature-Generator direkt in die Trainingsumgebung eingebunden. Dies ermöglichte es, schnell verschiedene Kreaturen zu testen. Die Bugreports und Featurerequests zum Generator werden direkt als Issue in das entsprechende GitHub Repository geschrieben und werden gegebenenfalls im Jour Fixe oder über Discord besprochen.


\subsection{Generierung der Terrain-Flächen}
Um ein Terrain mit natürlich erscheinenden Höhenunterschieden zu generieren, wird eine Perlin-Noise als Basis verwendet.
Hierbei entstehen sanfte Höhenunterschiede, die für eine kleine Wölbung des Terrains sorgen.
Mithilfe einer Perlin-Noise lassen sich auf das Terrain auch Flächen setzen, welche stärkere Höhenunterschiede haben.
Diese Idee wird verwendet um eine Begrenzung des Terrains in Form von Gebirgen am Rande des Terrains zu generieren.
Auch Hindernisse lassen sich mit diesem Ansatz generieren.

Allgemein wird das Platzieren der einzelnen Objekte wie Gebirge, Hindernisse, Vegetation oder Creature-Spawn-Points durch \emph{Zonen}\footnote{inspiriert von: \url{https://youtu.be/h9tLcD1r-6w?t=2416}} ermöglicht.
Diese Zonen bestimmen die Kategorie des platzierten Objektes an einer Koordinate und halten fest, ob überhaupt ein Objekt an einer Position gesetzt wurde.
Dadurch lässt sich auch bestimmen, welcher Bereich des Terrains noch frei ist, oder durch Objekte belegt wurde.
Dies ist hilfreich um festzulegen, wo beispielsweise zusätzliche Monster generiert werden können.
Auch lassen sich die Zonen nutzen um abhängig von der Umgebung der Zone, passende Vegetation zu generieren oder Texturen zu setzen.
Ein Vorteil dieses Ansatzes ist es, dass leicht neue Kategorien von Objekten eingeführt werden können und diese regelbasiert in die Umgebung eingefügt werden können.
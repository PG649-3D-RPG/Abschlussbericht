\section{Diskussion}
\label{Diskussion}

Dieser Abschnitt diskutiert die zuvor beschriebenen Ergebnisse der Projektgruppe im Bezug zur initialen Zielsetzung \ref{Zielsetzung_und_Vorgehensweise}.

\paragraph{Die Generierung von Spielleveln}
Die Ziele bezüglich der Generierung von Spielleveln wurden erfolgreich umgesetzt.
Das Layout des Spiellevels wird prozedural mittels Space-Partitioning generiert und anschließend durch Terrain Transformationen in eine natürlich wirkende Welt verwandelt.
Die Welt besteht aus Räumen und Korridoren, die durch Gebirge voneinander getrennt werden.
Anders als bei einer typischen Generierung von Dungeons mittels Space-Partitioning, wird hierbei auf die Decke des Dungeons verzichtet, sodass die Welt aus offenen Arenen besteht, die durch Korridore miteinander verbunden sind.
In der Welt werden Spawn-Punkte für den Spieler, Kreaturen und Hindernisse platziert.
Die Hindernisse bestehen aus Pflanzen, die zur Laufzeit aus L-Systemen erzeugt werden.
Allgemein ist es möglich die Größe der Welt, die Anzahl an Räumen und die Anzahl an Spawn-Punkten für Hindernisse und Kreaturen einzustellen.
Der Schwierigkeitsgrad lässt sich durch kleinere Räume mit mehr Spawn-Punkten erhöhen.

\paragraph{Generierung der Monster}

\paragraph{Fortbewegung der Monster} \fup

Das festgelegte minimal Ziel für die Fortbewegung der Monster war, dass die Animationen nicht manuell erstellt werden sollen. 
Stattdessen sollte mit Deep Reinforcement Learning ein Agent trainiert werden, der lernt die Monster zu bewegen, indem er Kräfte auf deren Joints ausübt.
Bei dem Versuch die ursprünglichen Vorstellung der Fortbewegung der Kreaturen umzusetzen traten allerdings einige Probleme auf.\\

Zum einen war das Trainieren der Fortbewegung von vollständig zufällige erstellten Kreaturen nicht möglich. 
In den ersten Trainingsdurchläufen wurde schnell klar, dass einige Kreaturen durch die Struktur ihres Körpers nicht dazu in der Lage waren sich stabil zu bewegen oder aufzustehen.
Ein Grund dafür kann zum Beispiel sein, dass die Bewegung einiger Joints zu weit eingeschränkt waren und die Agenten deswegen die mit diesen Joints verbundenen Körperteile nicht auf eine Art bewegen, welche eine stabile Fortbewegung ermöglicht. Das selbe Problem trat auf, wenn eine Kreatur am Boden lag und im Zweifel ihre Arme und Beine nicht genug Freiheit hatten, um aus dieser Position herauszukommen.
Das entgegengesetzte Problem konnte auch Auftreten, wenn die Freiheitsgrade zu hoch waren, da die Bewegungen dann nichts mehr mit der Fortbewegung von realen Tieren gemein hatten. Es war also notwendig die Parameter zur Generierung der Kreaturen einzuschränken, damit Fortbewegung möglich war.\\

Ein anderes Problem war die Stabilität der Fortbewegung. Die Vierbeinige Kreatur, deren Trainingsergebnisse in dem Abschnitt \ref{ErgebnisseTraining} vorgestellt werden, ist nach dem Training in der Lage stabil zu Laufen und auch wieder Aufzustehen und kann deswegen in dem Spiel verwendet werden. 
Dies war allerdings nicht für alle generierten Vierbeiner der Fall, selbst wenn die zuvor erwähnten Einschränkungen an den Generator übergeben werden. Nach dem Training konnten zwar fast alle Vierbeiner Laufen, aber in der Stabilität gab es große Variationen und einige der Kreaturen konnten nicht lernen aufzustehen. Aus diesem Grund wurde nur das eine vorgestellte Vierbeiner Modell dem Spiel hinzugefügt.\\
Die Zweibeinige Kreatur ist nach dem Absolvierten Training zwar in der Lage zu Laufen, aber dies ist insbesondere in Kurven sehr instabil. Dieses kann allerdings auch der gewählten Rewardfunktion geschuldet sein, da diese für das gesamte Training eine Geschwindigkeit vorgibt, welche der Agent einhalten sollten. Insbesondere in engen Kurven ist es für einen Zweibeiner aber aus Stabilitätsgründen effizient seine Geschwindigkeit zu verringern, wofür die Funktion den Agenten bestrafen würde. Das Problem mit der Stabilität ließe sich also eventuell entweder durch eine andere Rewardfunktion, die in Kurven langsamere Geschwindigkeiten erlaubt, oder eine andere Nav-Mesh Implementierung, die keine engen Kurven macht, lösen. Im Rahmen der Projektgruppe war allerdings keine Zeit mehr diese Vermutungen zu überprüfen.\\
Die Zweibeinern hatten außerdem Probleme mit dem Aufstehen. Nach dem Training konnte die Kreatur zwar aus dem Liegen wieder in den Stand kommen, aber sie blieb danach nicht stehen, sondern fiel wieder zu Boden. Diese Kombination aus instabilem Laufen mit häufigen Umfallen in Kurven und der Unfähigkeit aufzustehen macht die Zweibeiner ungeeignet für eine Verwendung in dem Spiel. Daher wurde entschieden keine Zweibeinigen Monster in das Spiel zu integrieren. \\

Abschließen lässt sich also sagen, dass das zu Beginn festgelegte minimal Ziel erreicht wurde und für die exemplarischen Kreaturen Modelle trainiert wurde, dass diese zum Laufen und Aufstehen verwenden können. Außerdem wurde eine einfacher Mechanismus implementiert, der Anhand einer übergebenen Bedingung entscheidet, ob die Kreatur im nächsten Schritt das Modell zum Laufen und das zum Aufstehen verwenden soll. \\
Die Umsetzung der weiterführenden Ideen ist allerdings zum größten Teil gescheitert. 
Insbesondere die Ansätze zur Diversifikation der Generierung, ein Agent der allen Kreaturen beibringen kann zu Laufen, und zur Generalisierung des Trainings, ein Modell welches von allen Kreaturen mit ähnlichen Skeletten zum Laufen verwendet wird, konnten nicht umgesetzt werden.

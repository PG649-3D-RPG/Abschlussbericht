\chapter{Evaluation}
\label{chap:evaluation}
In diesem Kapitel wird die Trainingsqualität der generierten Kreaturen evaluiert. 

\section{Einschränkungen der Evaluation}

\subsection{Auswahl der Kreaturen}
Die Evaluation wird aufgrund der limitierten Rechenressourcen auf jeweils eine Vierbeiner Kreatur und eine Zweibeiner Kreatur beschränkt. Abbildung \ref{fig:4B_creature_settings} zeigt die Parameter zur Generierung der Vierbeiner Kreatur. Abbildung \ref{fig:2B_creature_settings} zeigt die Parameter zur Generierung der Zweibeiner Kreatur.

\begin{figure}[ht]
    \centering
    \includegraphics[width=0.5\linewidth]{example-image-a}
    \caption{4B Parameter}\label{fig:4B_creature_settings}
\end{figure}

\begin{figure}[ht]
    \centering
    \includegraphics[width=0.5\linewidth]{example-image-a}
    \caption{2B Parameter}\label{fig:2B_creature_settings}
\end{figure}

\subsection{Konfiguration der Trainingsdurchläufe}
Die Erfahrung während der Entwicklungsphase der Projektgruppe hat gezeigt, dass die verwendete PPO Implementierung weitestgehend robust gegenüber der verwendeten Hyperparameter ist, solange eine ausreichend große Batchgröße verwendet wird. Aufgrund der limitierten Rechenressourcen wurde deswegen auf eine Hyperparameteroptimierung verzichtet. Abbildung \ref{fig:evaluation_config} zeigt die für die Evaluation verwendete neroRL Konfigurationsdatei. 
Für die Trainingsdurchläufe werden Unity Builds mit 16 Agenten verwendet, die auf 8 parallelen Workern ausgeführt werden. Dementsprechend werden die Daten parallel von 128 Agenten gesammelt. Jeder Agent führt bei jeder fünften Aktualisierung der Physikengine eine Aktion aus. Für das Training wird eine Batchgröße von 102400 verwendet, somit entsprechen 10 Updates etwa 1 Mio Schritten in der Umgebung.
Die in Abschnitt \ref{} beschriebenen Reward-Funktionen für das Aufstehen und Laufen werden jeweils zwei Mal für die Vierbeiner Kreatur und zwei Mal für die Zweibeiner Kreatur trainiert. Dabei werden alle 100 Updates die Policies zwischengespeichert und zum Sammeln der Evaluationsdaten verwendet. Für die Evaluation werden mit den gespeicherten Policies jeweils 128 zufällige Episoden ausgeführt und dabei Reward und Episodenlänge gesammelt.

\begin{figure}[ht]
    \centering
    \includegraphics[width=0.5\linewidth]{example-image-a}
    \caption{neroRL Konfigurationsdatei für die Evaluation}\label{fig:evaluation_config}
\end{figure}


\section{Ergebnisse}
In diesem Abschnitt werden die Ergebnisse der Trainingsdurchläufe für die Vierbeiner und Zweibeiner Kreaturen beschrieben. Zunächst werden Daten bezüglich des Rewards und der Länge der Trainingsepisoden analysiert. Anschließend wird die Qualität der Trainingsergebnisse bei der visuellen Evaluation in Unity beschrieben.

\subsection{Vierbeiner Kreatur}

\subsubsection{Laufen}
Die Entwicklung des Rewards beim Trainieren des Laufens der Vierbeiner Kreatur ist in Abbildung \ref{fig:Walking4B_Reward} abgebildet. Der Reward steigt in den ersten 1000 Updates deutlich an und erreicht ein Maximum von 14000. Danach fällt der Reward langsam ab auf ca. 7000-1000, ein Policy Collapse tritt nicht ein. 

\begin{figure}[ht]
    \centering
    \includegraphics[width=0.5\linewidth]{resources/img/results/Walking4B_Reward.png}
    \caption{Walking 4B Reward}\label{fig:Walking4B_Reward}
\end{figure}

Die Episodenlänge der Vierbeiner beim Laufen ist unbeschränkt, eine Episode wird beendet, wenn die Kreatur mit einem Körperteil außer den Füßen den Boden berührt. Da alle 5 Physik-Aktualisierungen der Unity Umgebung eine Aktion des Agenten angefordert wird und die Unity Umgebung mit 120 Aktualisierungen pro Sekunde ausgeführt wird, entspricht eine Länge von 1000 ca. 41.5 Sekunden. Abbildung \ref{fig:Walking4B_Length} zeigt die Entwicklung der Episodenlänge während der Trainingsdurchläufe. Analog zum Reward steigt die durchschnittliche Episodenlänge in den ersten 1000 Updates deutlich an und stagniert danach bzw. fällt langsam ab. Das Maximum wird mit einer Episodenlänge von ca. 6000, umgerechnet ca. 4 Minuten und 9 Sekunden, erreicht.

\begin{figure}[ht]
    \centering
    \includegraphics[width=0.5\linewidth]{resources/img/results/Walking4B_Length.png}
    \caption{Walking 4B Length}\label{fig:Walking4B_Length}
\end{figure}

Die Visuelle Evaluation in Unity zeigt, dass die Vierbeiner Kreatur mit beliebigen Policies ab ca. 500 Updates weitestgehend stabil läuft. Instabilität tritt insbesondere dann auf, wenn die Kreatur mit hoher Geschwindigkeit einen Wegpunkt erreicht und eine große Drehung in Richtung des nächsten Wegpunkts durchführen muss. 

\subsubsection{Aufstehen}
Abbildung \ref{fig:Standup4B_Reward} zeigt die Entwicklung des Rewards während der Updates des Trainingsdurchlaufs. Der Reward steigt in den ersten 250 Updates deutlich von 0 auf ca. 1700. Das Training stagniert für mehrere hundert Updates bei ca. 1700, bevor ein Performance Collapse eintritt, von dem sich die Policy nicht erneut erholt.

\begin{figure}[ht]
    \centering
    \includegraphics[width=0.5\linewidth]{resources/img/results/Standup4B_Reward.png}
    \caption{Standup 4B Reward}\label{fig:Standup4B_Reward}
\end{figure}

Die Episodenlänge für das Aufstehen der Vierbeiner ist auf 1000 Schritte beschränkt und es gibt keine Bedingung, die den Start einer neuen Episode vor Ablauf der Schritte bedingt. Da alle 5 Schritte eine Aktion des Agenten angefordert wird, liegen die in Abbildung \ref{fig:Standup4B_Length} dargestellten Episodenlängen konstant bei 201 Schritten.

\begin{figure}[ht]
    \centering
    \includegraphics[width=0.5\linewidth]{resources/img/results/Standup4B_Length.png}
    \caption{Standup 4B Length}\label{fig:Standup4B_Length}
\end{figure}

Die Visuelle Evaluation in Unity zeigt, dass beliebige Policies aus den Update-Bereichen, in denen der Reward bei ca. 1700 stagniert, in der Lage sind den Vierbeiner erfolgreich aus einer liegenden Startposition aufstehen zu lassen. Die Policies nach dem Performance Collapse führen nur minimal wahrnehmbare Aktionen aus und die Kreatur bleibt unbewegt liegen. 

\subsection{Zweibeiner Kreatur}

\subsubsection{Laufen}
Abbildung \ref{fig:Walking2B_Reward} zeigt die Entwicklung des Rewards beim Trainieren des Laufens der Zweibeiner Kreatur. Der Reward steigt über die ersten ca. 1500 Updates von 0 auf ca. 3000-4000 deutlich an. Danach ist steigt der Reward langsam weiter und schwankt dabei stark. In den Trainingsdurchläufen der Evaluation wurde ein maximaler Reward von ca. 5300 erreicht.

\begin{figure}[ht]
    \centering
    \includegraphics[width=0.5\linewidth]{resources/img/results/Walking2B_Reward.png}
    \caption{Walking 2B Reward}\label{fig:Walking2B_Reward}
\end{figure}

Die Episodenlänge der Zweibeiner beim Laufen ist nicht beschränkt. Eine Episode wird beendet, wenn die Kreatur mit einem Körperteil außer den Füßen den Boden berührt. Die in Abbildung \ref{fig:Walking2B_Length} dargestellte Episodenlänge steigt parallel zum Reward in den ersten 1500 Updates deutlich von 0 auf ca. 1000 und stagniert danach mit Schwankung. Da alle 5 Schritte eine Aktion des Agenten angefordert wird, bedeutet dies ca. 5000 Schritte. Die maximale durchschnittliche Episodenlänge beträgt ca. 2000.

% TODO Ab hier hat Nils dran rumgepfuscht 
\begin{figure}[ht]
    \centering
    \includegraphics[width=0.5\linewidth]{resources/img/results/Walking2B_Length.png}
    \caption{Walking 2B Length}\label{fig:Walking2B_Length}
\end{figure}

Die Visuelle Evaluation in Unity zeigt, dass die Policies mit höchstem Reward den Zweibeiner laufen lassen. In Abbildung \ref{fig:2BLaufen} wird ein Laufschritt auf gerade ebene mit einer Geschwindigkeit von $5$ dargestellt.
\begin{figure}
	\centering
	\begin{subfigure}[b]{0.3\textwidth}
		\centering
		\includegraphics[width=\textwidth]{resources/img/Unity1}
		\caption{Auftreten}
		\label{fig:Laufen1}
	\end{subfigure}
	\hfill
	\begin{subfigure}[b]{0.3\textwidth}
		\centering
		\includegraphics[width=\textwidth]{resources/img/Unity2}
		\caption{Schritt}
		\label{fig:Laufen2}
	\end{subfigure}
	\hfill
	\begin{subfigure}[b]{0.3\textwidth}
		\centering
		\includegraphics[width=\textwidth]{resources/img/Unity3}
		\caption{Landen}
		\label{fig:Laufen3}
	\end{subfigure}
	\caption{3 Schritte im Laufzyklus eines Zweibeiners.}
	\label{fig:2BLaufen}
\end{figure}

Hier ist zu erkennen, dass es sich nicht um ein Laufen wie bei einem Menschen, welcher ein Fuß auf den Boden stehen hat, den anderen nach vorne zieht und später den zweiten hinter-herzieht, handelt. Sondern es eher einem Rennen gleicht, bei dem beide Beine sich in der Luft befinden. Die Belohnung für das Erreichen der Geschwindigkeit wird hierbei nicht vollständig erreicht und schwankt im Bereich zwischen $0.5$ und $0.6$\footnote{Hier wurden nur die Werte aus dem Log der Belohnungfunktion im Editor abgelesen. Es kann sein, dass die Stichprobe eine Ausnahme abbildet, obwohl diese über mehrere Resets gleich geblieben ist. Eine genauere Untersuchung wäre als Folgearbeit angeraten.}.  Ein weniger trainiert Netzwerk\footnote{Zum Sichttest wurden die Netzwerke \texttt{Generated2B-2000} und \texttt{Generated2B-3000} benutzt. Das erstere ist hierbei das weniger trainierte Netzwerk.} 

Das Laufen ist insbesondere in Kurven instabil. In Abbildung \ref{fig:2bKurve} ist eine Kreatur in einer Kurve abgebildet. Hierbei ist die rote Linie der Pfad an Kontrollpunkte zum Ziel. Der jeweils nächste Punkt ist die Kugel und bildet das Ziel des Agenten ab. Da die Figur mit einer relativ hohen Geschwindigkeit in die Kurve geht und die Stabilität des Laufens dafür nicht ausreicht, springt dieses Ziel häufig in Kurven und trägt so zu der erhöhten Instabilität in Kurven bei.

\begin{figure}
	\centering
	\includegraphics[width=0.7\linewidth]{resources/img/Unity_id9Hnh5u8N}
	\caption[Zweibeiner in einer Kurve]{Die Stabilitätsprobleme eines Zweibeiners in einer Kurve. Hierbei ist die Kugel das aktuell anvisierte Ziel und die rote Linie die Kurve der weiteren Ziele.}
	\label{fig:2bKurve}
\end{figure}

\subsubsection{Aufstehen}

Abbildung \ref{fig:Standup2B_Reward} zeigt die Entwicklung des Rewards während der Updates des Trainingsdurchlaufs. Der Reward steigt in den ersten 1500 bis 2000 Updates kontinuierlich von 0 auf ca. 1200. Danach tritt ein Performance Collapse ein, von dem sich die Policy innerhalb der weiteren berechneten 2000 Updates nur langsam erholt.


\begin{figure}[ht]
    \centering
    \includegraphics[width=0.5\linewidth]{resources/img/results/Standup2B_Reward.png}
    \caption{Standup 2B Reward}\label{fig:Standup2B_Reward}
\end{figure}

Die Episodenlänge für das Aufstehen der Zweibeiner ist auf 1000 Schritte beschränkt und es gibt keine Bedingung, die den Start einer neuen Episode vor Ablauf der Schritte bedingt. Da alle 5 Schritte eine Aktion des Agenten angefordert wird, liegen die in Abbildung \ref{fig:Standup2B_Length} dargestellten Episodenlängen konstant bei 201 Schritten.

\begin{figure}[ht]
    \centering
    \includegraphics[width=0.5\linewidth]{resources/img/results/Standup2B_Length.png}
    \caption{Standup 2B Length}\label{fig:Standup2B_Length}
\end{figure}

Die Visuelle Evaluation in Unity zeigt, dass die besten Policies (nach 1400 und 1800 Updates) die Zweibeiner Kreatur schwungvoll aufstehen lassen, die Kreatur aber nicht in der aufrechten Position halten können, sodass die Kreatur erneut hinfällt. Die Policies nach dem Performance Collapse führen nur minimal wahrnehmbare Aktionen aus und die Kreatur bleibt unbewegt liegen. 


\section{Diskussion}
\label{Diskussion}

Dieser Abschnitt diskutiert die zuvor beschriebenen Ergebnisse der Projektgruppe im Bezug zur initialen Zielsetzung \ref{Zielsetzung_und_Vorgehensweise}.

\paragraph{Die Generierung von Spielleveln}
Die Ziele bezüglich der Generierung von Spielleveln wurden erfolgreich umgesetzt.
Das Layout des Spiellevels wird prozedural mittels Space-Partitioning generiert und anschließend durch Terrain Transformationen in eine natürlich wirkende Welt verwandelt.
Die Welt besteht aus Räumen und Korridoren, die durch Gebirge voneinander getrennt werden.
Anders als bei einer typischen Generierung von Dungeons mittels Space-Partitioning, wird hierbei auf die Decke des Dungeons verzichtet, sodass die Welt aus offenen Arenen besteht, die durch Korridore miteinander verbunden sind.
In der Welt werden Spawn-Punkte für den Spieler, Kreaturen und Hindernisse platziert.
Die Hindernisse bestehen aus Pflanzen, die zur Laufzeit aus L-Systemen erzeugt werden.
Allgemein ist es möglich die Größe der Welt, die Anzahl an Räumen und die Anzahl an Spawn-Punkten für Hindernisse und Kreaturen einzustellen.
Der Schwierigkeitsgrad lässt sich durch kleinere Räume mit mehr Spawn-Punkten erhöhen.

\paragraph{Generierung der Monster}
Von Anfang an wurden bei der Generierung der Monster viele methodische Freiheiten gelassen. Damit konnten Beschränkungen bezüglich der Körperausprägungen vermieden werden. Während der Ausarbeitung sind dann die Zwei- und Vierbeiner in den Vordergrund gerückt, da diese auch für das Training bekannte und einfach erweiterbare Ansätze ermöglicht haben. Gleichzeitig ist schnell ersichtlich gewesen, dass die Auswahl der Körperteile eingeschränkt werden sollte, sodass die Komplexität der Kreaturen auf ein Minimum beschränkt wird und möglichst wenig Freiheitsgrade für das Training und Animieren der Kreaturen notwendig sind. Trotzdem ist es möglich, die entstandene parametrische Methode von Jona Heinrichs so zu erweitern, dass mit möglichst wenig Aufwand, Kreaturen mit einem noch größer variierenden Körperbau (z.B. Flügel, noch mehr Gliedmaßen, Acht-Füßler, etc.) entstehen können. Bei der parametrischen Methode ist es weiterhin wichtig, dass die Wertebereiche der Parameter passend zu der erwartenden Anatomie eines Zwei- und/oder Vierbeiners gewählt werden und dieser sich physikalisch korrekt animieren lässt.

Somit sind die Ziele, dass die Kreaturen-Erstellung Algorithmen-basiert ist und die innerhalb von Unity bereitgestellten Frameworks (z.B. Joints) verwendet werden sollten, erreicht worden. Für das Erreichen dieser Ziele, wurde sich ebenfalls so wie Anfangs angeschnitten, erfolgreich an dem Unity-ML-Agents-Walker orientiert.

Die folgenden Darstellungen in \ref{fig:finished_CG_2} \& \ref{fig:finished_CG_4} repräsentieren das fertige Ergebnis der Creature-Generator.

\begin{figure}[ht]
    \centering
    \begin{subfigure}[b]{0.2\textwidth}
        \centering        
        \includegraphics[width=\textwidth, height=\textwidth]{resources/img/Finished_Creatures_2/creature_1}
    \end{subfigure}
    \begin{subfigure}[b]{0.2\textwidth}
        \centering
        \includegraphics[width=\textwidth, height=\textwidth]{resources/img/Finished_Creatures_2/creature_2}
    \end{subfigure}
    \begin{subfigure}[b]{0.2\textwidth}
        \centering        
        \includegraphics[width=\textwidth, height=\textwidth]{resources/img/Finished_Creatures_2/creature_3}
    \end{subfigure}
    \begin{subfigure}[b]{0.2\textwidth}
        \centering
        \includegraphics[width=\textwidth, height=\textwidth]{resources/img/Finished_Creatures_2/creature_4}
    \end{subfigure}
    \begin{subfigure}[b]{0.2\textwidth}
        \centering        
        \includegraphics[width=\textwidth, height=\textwidth]{resources/img/Finished_Creatures_2/creature_5}
    \end{subfigure}
    \begin{subfigure}[b]{0.2\textwidth}
        \centering
        \includegraphics[width=\textwidth, height=\textwidth]{resources/img/Finished_Creatures_2/creature_6}
    \end{subfigure}
    \begin{subfigure}[b]{0.2\textwidth}
        \centering
        \includegraphics[width=\textwidth, height=\textwidth]{resources/img/Finished_Creatures_2/creature_7}
    \end{subfigure}
    \begin{subfigure}[b]{0.2\textwidth}
        \centering
        \includegraphics[width=\textwidth, height=\textwidth]{resources/img/Finished_Creatures_2/creature_8}
    \end{subfigure}
    \begin{subfigure}[b]{0.2\textwidth}
        \centering
        \includegraphics[width=\textwidth, height=\textwidth]{resources/img/Finished_Creatures_2/creature_9}
    \end{subfigure}
    \begin{subfigure}[b]{0.2\textwidth}
        \centering
        \includegraphics[width=\textwidth, height=\textwidth]{resources/img/Finished_Creatures_2/creature_10}
    \end{subfigure}
    \begin{subfigure}[b]{0.2\textwidth}
        \centering
        \includegraphics[width=\textwidth, height=\textwidth]{resources/img/Finished_Creatures_2/creature_11}
    \end{subfigure}
    \begin{subfigure}[b]{0.2\textwidth}
        \centering
        \includegraphics[width=\textwidth, height=\textwidth]{resources/img/Finished_Creatures_2/creature_12}
    \end{subfigure}
    \caption{Der fertige Creature-Generator Stand: Beispiele für Zweibeiner}
    \label{fig:finished_CG_2}
\end{figure}

\begin{figure}[ht]
    \centering
    \begin{subfigure}[b]{0.2\textwidth}
        \centering        
        \includegraphics[width=\textwidth, height=\textwidth]{resources/img/Finished_Creatures_4/creature_1}
    \end{subfigure}
    \begin{subfigure}[b]{0.2\textwidth}
        \centering
        \includegraphics[width=\textwidth, height=\textwidth]{resources/img/Finished_Creatures_4/creature_2}
    \end{subfigure}
    \begin{subfigure}[b]{0.2\textwidth}
        \centering        
        \includegraphics[width=\textwidth, height=\textwidth]{resources/img/Finished_Creatures_4/creature_3}
    \end{subfigure}
    \begin{subfigure}[b]{0.2\textwidth}
        \centering
        \includegraphics[width=\textwidth, height=\textwidth]{resources/img/Finished_Creatures_4/creature_4}
    \end{subfigure}
    \begin{subfigure}[b]{0.2\textwidth}
        \centering        
        \includegraphics[width=\textwidth, height=\textwidth]{resources/img/Finished_Creatures_4/creature_5}
    \end{subfigure}
    \begin{subfigure}[b]{0.2\textwidth}
        \centering
        \includegraphics[width=\textwidth, height=\textwidth]{resources/img/Finished_Creatures_4/creature_6}
    \end{subfigure}
    \begin{subfigure}[b]{0.2\textwidth}
        \centering
        \includegraphics[width=\textwidth, height=\textwidth]{resources/img/Finished_Creatures_4/creature_7}
    \end{subfigure}
    \begin{subfigure}[b]{0.2\textwidth}
        \centering
        \includegraphics[width=\textwidth, height=\textwidth]{resources/img/Finished_Creatures_4/creature_8}
    \end{subfigure}
    \begin{subfigure}[b]{0.2\textwidth}
        \centering
        \includegraphics[width=\textwidth, height=\textwidth]{resources/img/Finished_Creatures_4/creature_9}
    \end{subfigure}
    \begin{subfigure}[b]{0.2\textwidth}
        \centering
        \includegraphics[width=\textwidth, height=\textwidth]{resources/img/Finished_Creatures_4/creature_10}
    \end{subfigure}
    \begin{subfigure}[b]{0.2\textwidth}
        \centering
        \includegraphics[width=\textwidth, height=\textwidth]{resources/img/Finished_Creatures_4/creature_11}
    \end{subfigure}
    \begin{subfigure}[b]{0.2\textwidth}
        \centering
        \includegraphics[width=\textwidth, height=\textwidth]{resources/img/Finished_Creatures_4/creature_12}
    \end{subfigure}
    \caption{Der fertige Creature-Generator Stand: Beispiele für Vierbeiner}
    \label{fig:finished_CG_4}
\end{figure}


\paragraph{Fortbewegung der Monster} \fup

Das festgelegte Minimalziel für die Fortbewegung der Monster war, dass die Animationen nicht manuell erstellt werden sollen. 
Stattdessen sollte mit Deep Reinforcement Learning ein Agent trainiert werden, der lernt die Monster zu bewegen, indem er Kräfte auf deren Joints ausübt.
Bei dem Versuch die ursprünglichen Vorstellung der Fortbewegung der Kreaturen umzusetzen traten allerdings einige Probleme auf.\\

Zum einen war das Trainieren der Fortbewegung von vollständig zufällig erstellten Kreaturen nicht möglich. 
In den ersten Trainingsdurchläufen wurde schnell klar, dass einige Kreaturen durch die Struktur ihres Körpers nicht dazu in der Lage waren sich stabil zu bewegen oder aufzustehen.
Ein Grund dafür kann zum Beispiel sein, dass die Bewegung einiger Joints zu weit eingeschränkt sind und die Agenten deswegen die mit diesen Joints verbundenen Körperteile nicht auf eine Art bewegen können, welche eine stabile Fortbewegung ermöglicht. Das selbe Problem trat auf, wenn eine Kreatur am Boden lag und im Zweifel ihre Arme und Beine nicht genug Freiheit hatten, um aus dieser Position herauszukommen.
Das entgegengesetzte Problem konnte auch auftreten, wenn die Freiheitsgrade zu hoch waren, da die Bewegungen dann nichts mehr mit der Fortbewegung von realen Tieren gemein hatten. Es war also notwendig die Parameter zur Generierung der Kreaturen einzuschränken, damit Fortbewegung möglich war.\\

Ein anderes Problem war die Stabilität der Fortbewegung. Die vierbeinige Kreatur, deren Trainingsergebnisse in dem Abschnitt \ref{ErgebnisseTraining} vorgestellt werden, ist nach dem Training in der Lage stabil zu laufen und auch wieder aufzustehen und kann deswegen in dem Spiel verwendet werden. 
Dies war allerdings nicht für alle generierten Vierbeiner der Fall, selbst wenn die zuvor erwähnten Einschränkungen an den Generator übergeben werden. Nach dem Training konnten zwar fast alle Vierbeiner laufen, aber in der Stabilität gab es große Variationen und einige der Kreaturen konnten nicht lernen aufzustehen. Aus diesem Grund wurde nur das eine vorgestellte Vierbeiner Modell dem Spiel hinzugefügt.\\
Die zweibeinige Kreatur ist nach dem absolvierten Training zwar in der Lage zu Laufen, aber dies ist insbesondere in Kurven sehr instabil. Dieses kann allerdings auch der gewählten Rewardfunktion geschuldet sein, da diese für das gesamte Training eine Geschwindigkeit vorgibt, welche der Agent einhalten sollte. Insbesondere in engen Kurven ist es für einen Zweibeiner aber aus Stabilitätsgründen effizient seine Geschwindigkeit zu verringern, wofür die Funktion den Agenten bestrafen würde. Das Problem mit der Stabilität ließe sich also eventuell entweder durch eine andere Rewardfunktion, die in Kurven langsamere Geschwindigkeiten erlaubt, oder eine andere Nav-Mesh Implementierung, die keine engen Kurven macht, lösen. Im Rahmen der Projektgruppe war allerdings keine Zeit mehr diese Vermutungen zu überprüfen.\\
Die Zweibeiner hatten außerdem Probleme mit dem aufstehen. Nach dem Training konnte die Kreatur zwar aus dem liegen wieder in den Stand kommen, aber sie blieb danach nicht stehen, sondern fiel wieder zu Boden. Diese Kombination aus instabilem Laufen mit häufigen umfallen in Kurven und der Unfähigkeit aufzustehen macht die Zweibeiner ungeeignet für eine Verwendung in dem Spiel. Daher wurde entschieden keine zweibeinigen Monster in das Spiel zu integrieren. \\

Abschließend lässt sich also sagen, dass das zu Beginn festgelegte Minimalziel erreicht wurde und für die exemplarischen Kreaturen Modelle trainiert wurden, dass diese zum laufen und aufstehen verwenden können. Außerdem wurde ein einfacher Mechanismus implementiert, der Anhand einer übergebenen Bedingung entscheidet, ob die Kreatur im nächsten Schritt das Modell zum laufen und das zum aufstehen verwenden soll. \\
Die Umsetzung der weiterführenden Ideen ist allerdings zum größten Teil gescheitert. 
Insbesondere die Ansätze zur Diversifikation der Generierung, ein Agent der allen Kreaturen beibringen kann zu laufen, und zur Generalisierung des Trainings, ein Modell welches von allen Kreaturen mit ähnlichen Skeletten zum laufen verwendet wird, konnten nicht umgesetzt werden.

\paragraph{Handlungsempfehlungen}
Die Ergebnisse der Projektgruppe zeigen, dass das Generieren von Bewegungsanimationen für prozedural generierte Skelette durch Reinforcement Learning möglich ist. Für die Zwei- und Vierbeiner Kreaturen werden vielversprechende Eregbnisse erzielt, wobei insbesondere die Ergebnisse der Vierbeiner ohne größere Einschränkungen im entwickelten Spiel einsetzbar sind.
Allerdings unterliegen die Ergebnisse diversen Einschränkungen, die in den vorherigen Abschnitten detailliert beschrieben wurden. Diese Einschränkungen stehen teilweise im Widerspruch zu den initialen Erwartungen, dass Reinforcement Learning basierte Methoden genereller und mit geringerem Aufwand einsetzbar sind, als traditionelle Keyframe Methoden zur Charakteranimation. Der Aufbau einer Umgebung, die die Anwendung von Reinforcement Learning erlaubt, ist mit großem Aufwand verbunden. Die Umgebung muss die Zielumgebung (d.h. das tatsächliche Spielumfeld der Kreaturen) so modellieren, dass die gelernten Bewegungsanimationen auch im fertigen Spiel einsetzbar sind. Um realistische Animationen zu trainieren, die an die Bewegung echter Lebewesen erinnern, müssen entsprechende Einschränkungen in die Rewardfunktion integriert werden. Die Modellierung dieser Einschränkungen erfordert tiefgehendes Wissen über die zu trainierenden Kreaturen, um beispielsweise auf Positionen der Knochen zuzugreifen. Außerdem ist die Entwicklung einer funktionierenden Rewardfunktion nicht trivial, da die tatsächlichen Effekte der einzelnen Rewards nicht eindeutig sind. Somit vereinfacht die hier beschriebene Anwendung von Reinforcement Learning zur Charakteranimation diese nur bedingt.
Sind die Einschränkungen und initialen Hindernisse überwunden und es existieren eine Modellumgebung, die für das Training verwendet werden kann, sowie Rewardfunktionen für die gewünschten Bewegungen, können durch Reinforcement Learning relativ schnell Animationen für zusätzliche Charaktere generiert werden. Je nach Komplexität der Umgebung und der Skelette, und abhängig von der verfügbaren Hardware, dauert ein Trainingsprozess trotzdem mehrere Tage, sodass potenziell auch Keyframe Animationen mit ähnlichem Zeitaufwand erstellt werden könnten. Die Generierung von Animationen für neue Kreaturen in Echtzeit ist nicht möglich. Es ist außerdem nicht garantiert, dass neu generierte Kreaturen mit den definierten Rewardfunktionen einen vergleichbaren Lernerfolg haben. Selbst mit den starken Einschränkungen bei der Generierung neuer Kreaturen im Rahmen der Projektgruppe, erzielt nicht jede Kreatur einen vergleichbaren Lernerfolg. Um das Problem der Laufzeit der Trainingsdurchläufe zu umgehen und verschiedenartige Kreaturen zu unterstützen, bietet sich ein generalisiertes Training an, dass Bewegungsanimationen für verschiedenartige Kreaturen lernt. Damit könnten sogar in Echtzeit neue Kreaturen generiert und animiert werden. Die Versuche zur Generalisierung während dieser Projektgruppe führten aufgrund der hohen Komplexität jedoch nicht zu erfolgreichen Ergebnissen.

Die Charakteranimation durch Reinforcement Learning mit den in diesem Bericht beschriebenen Methoden erzielt interessante und teilweise vielversprechende Methoden. Für Einsatzgebiete, die flexible und unrealistische Animationen (z.B. für die Animation von Monstern), den initialen Entwicklungsaufwand und die laufenden Hardwarekosten erlauben, lassen sich durch Reinforcement Learning vielfältige Animationen generieren.



\section{Probleme}
Bei der Bearbeitung der gegebenen Aufgabenstellung der Projektgruppe haben einige Probleme zu Verzögerungen geführt, welche sich auf die endgültige Funktionalität des Endprodukts ausgewirkt haben.

\subsection{Stabilität des Skeletts}
Das Hauptproblem für die Animatoren-Teilgruppe, welche Hauptsächlich mithilfe von maschinellem Training eine beliebige Kreatur zum laufen bringt, liegt in der Kreaturenstabilität. Zum Anfang der Arbeitsphase wurde die bestehende Walker-Umgebung von Unity analysiert und auf Basis der in dieser Umgebung vorhandenen Kreatur die ersten Tests erstellt. Hierbei konnte verifiziert werden, dass die neu entwickelte Trainingsumgebung funktioniert und die grundlegenden Einstellung zu funktionierenden Ergebnissen führen. Als danach die ersten prozedural generierten Kreaturen eingesetzt wurden, kam es zu einer Vielzahl von Problemen, welche durch andere Defizite in verschiedene Bereichen verstärkt wurden.

\subsubsection{Dokumentation} % Nils
Die meisten Projektgruppenmitglieder hatten vor der PG wenig Erfahrungen mit Unity. Deshalb ist die Dokumentation von Unity eine der wichtigsten Quellen für die Umsetzung der einzelnen Teilprojekte. Die Unity-Dokumentation\footnote{\url{https://docs.unity3d.com/Manual/index.html}} ist Online frei einsehbar für die verschiedenen Versionen der Grafik-Engine. Dabei ist problematisch, dass insbesondere in den Teil zur Physik-Engine oder neueren Pakete ist, welche noch nicht den Vorschau-Status verlassen haben, deutliche Formulierungen fehlen. Ein Beispiel dafür ist die Hilfestellung zur Ragdoll-Stabilität. In einen Nebensatz\footnote{Zu finden auf dieser Unterseite der Dokumentation \url{https://docs.unity3d.com/Manual/RagdollStability.html}} wird erwähnt, dass ein zu großer Massenunterschied zwischen zwei direkt verbundenen Elementen zu unruhigen Ragdolls führen kann. In der Praxis bedeutet dies, dass die generierten Kreaturen bei jegliche Krafteinwirkung explodieren. Eine Fehler-findung und -behebung dieses Problems hat mehrere Wochen gedauert, da die Auswirkungen nur in sehr abgeschwächter Form beschrieben wurden.

Ein weiteres Beispiel ist der \emph{Solver Type}\footnote{Die Komponente der Physikumgebung, welche die Berechnungen für die Kollisionserkennung durchführt.}, welche auf \emph{Projected Gauss Seidel} oder \emph{Temporal Gauss Seidel} gesetzt werden kann. Für das Training wurde in der Anfangsphase versucht jeweils die besten Einstellungen von Unity zu verwenden, was nach Dokumentation die zweitere Option sein sollte. In Gegensatz zu der Dokumentation warnen mehrere Internetquellen\footnote{Siehe beispielsweise das zum PG-Zeitpunkt \href{https://www.youtube.com/watch?v=aZ1zc6zZ61E}{erste Google-Ergebnis}} vor dieser Einstellung. Ein nicht repräsentativer Test hat dies für unsere Kreaturen bestätigt, weshalb im weiteren Verlauf \emph{Projected Gauss Seidel} genutzt wurde. Ein Hinweis, dass \emph{Temporal Gauss Seidel} problematische Ergebnisse produzieren kann, fehlt zum Zeitpunkt des Projetgruppenberichts weiterhin in der Dokumentation.

Insgesamt gab es weitere Beispiele, wie zum Beispiel bei dem \texttt{com.unity.ai.navigation}-Paket \footnote{Eine Dokumentation ist \href{https://docs.unity3d.com/Packages/com.unity.ai.navigation@1.0/manual/NavMeshSurface.html}{hier} zu finden} bei welchem die unterstützten Unity-Versionen unklar ist, welche zusammen zu viel Recherchearbeit geführt haben und deshalb die Bearbeitung der Kernaufgaben verzögert haben.

\subsubsection{Organisatorische Probleme}
Ein weiterer Teilbereich, der zu Verzögerung in den Arbeitsablauf der Animatorenteilgruppe geführt hat, ist organisatorischen Problemen zu zuschrieben. Einige der größten Hindernisse sind die starke Abhängigkeit von den Animatoren und Generatoren, veralte Rechenhardware und fehlender Vorkenntnisse.

Das erste Problem kann wie folgt beschrieben werden. Immer wenn die Änderung an der Kreatur nötig waren, musste das für die Generierung verantwortliche Paket angepasst werden. Inklusive der Kommunikation und der dafür benötigten Arbeitszeit dauerte dies ungefähr eine Woche. In diesen Phasen konnte das Training häufig nicht fortgesetzt werden, da die Kreaturen zu große Fehler hatten. In die andere Richtung konnten die Generatoren nicht weiterarbeiten, da diese auf Feedback von den Trainingsversuchen gewartet haben.

Verstärkt wurde dies durch das zweite Problem. Da LIDO von der ganzen Universität genutzt wird, kann es einige Stunden bis Tage dauern, bis eine Aufgabe abgearbeitet wird. Inklusive der Berechnungszeit des Auftrags konnten so 2 aufeinanderfolgende Experimente gestartet werden je Woche. Da insbesondere in der Mitte der Projektgruppe die Fehler nicht bekannt waren, dauerte das Finden dieser dadurch besonders lange. 
Zusätzlich zu der Wartezeit ist die Rechendauer eines Auftrags auf LIDO für Studenten eingeschränkt. Ein Knoten mit Grafikkarte kann 2 Tage lang reserviert werden. Bei den finalen Training auf einen Knoten des Lehrstuhls stellt sich heraus, dass diese Zeit nicht ausreichend ist, um das (lokale) Maximum der Belohnungsfunktion zu erreichen. Theoretisch wäre ein Fortsetzen der Trainingsaufgabe möglich, führt aber zu eine weiteren Wartezeit auf einen Knoten und Verfälschungen der Trainingsergebnisse durch nicht perfekt zurückgesetzten Parameter des Trainings. 
Des Weiteren stellte sich heraus, dass das Training auf den alten Knoten signifikant länger dauert, als auf den aktuell ausgestatteten Rechenknoten des Lehrstuhls. Ein Trainingsschritt auf den öffentlichen Knoten dauert um die $160 \si{\sec}$  und auf den Lehrstuhlknoten $60 \si{\sec}$. Dies führt zu einer 3-Fachen Wartedauer während der meisten Experimente.

Zuletzt fehlten insbesondere bei dem maschinellen Lernen viele Vorkenntnisse. Das zu Beginn der PG gehaltene Seminar beschäftigte sich mit den eigentlich genutzten Algorithmen, hat aber keine Übersicht über die Forschung im Bereich des physikalischen Laufens gegeben. Hier wurde später \cite{Geijtenbeek2012} genutzt, welches aufgrund des Erscheinungsjahrs kein Überblick für Netzwerkbasierte-Lernmethoden gibt. Eine Einschätzung von weiteren Papieren war dadurch erschwert. Weiterhin beziehen sich viele Arbeiten auf andere Physikumgebgungen, nutzen Imitation zum Lernen oder nutzen explizite Designcharakteristiken der Kreaturen aus\cite{Mourot2022}.

Insgesamt führten die Probleme häufig zu Arbeitsphasen in den die Animatorengruppe oder Generatoren keine neuen Ergebnisse produzieren konnten. In dieser Zeit wurde versucht zukünftige Aufgaben wie beispielsweise eine Generalisierung der Belohnungsfunktionen, eine stärkere anpassbare Trainingsumgebung oder Zusatzfunktionen wie ein unebener Boden. Durch die grundlegenden Probleme bei der Stabilität konnte am Ende keine dieser Erweiterungen fertiggestellt werden.


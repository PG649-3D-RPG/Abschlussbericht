\chapter{Fazit \& Ausblick}
\label{Ausblick}

Innerhalb der Projektgruppe wurden anfangs Pläne und Ziele angeführt, welche während der Arbeit allgegenwärtig waren und auf welchen die Umsetzung und Entwicklung auf- und ausgebaut wurde. Zwei Semester, in welchen über die Projektgruppe hinaus viel Zeit der Studierenden für andere Aspekte des Studiums hingegeben wurde, reichen nicht aus um einen fertigen Prototypen und erst recht nicht ein fertiges Spiel zu implementieren. Es mussten wie in der Evaluation diskutiert wurde daher mehrere Inhalte gekürzt werden. Zuletzt waren es die vierbeinigen Kreaturen welche erfolgreich die Skills des Aufstehens und des Laufens gelernt haben, wobei die Zweibeiner Inkonsistenzen aufgewiesen haben. Bestrebungen sollten von Anfang an realistisch gehalten werden, um zeitlich und inhaltlich nicht über die tatsächlichen Gegebenheiten hinaus zu laufen und damit die anfangs angeführten Pläne mitten in der Entwicklungsphase nochmal überarbeiten zu müssen. An dieser Stelle ergibt sich, dass die ursprünglichen Ziele in dieser Arbeit zu größten Teilen eingehalten wurden und vor allem relevant für den Entwicklungsprozess der Projektgruppe waren.

Insgesamt könnten weitere Maßnahmen ergriffen werden, um über die Entwicklung eines Prototypen hinaus zu einem ausgefeiltem Spiel zu gehen.

In der Einleitung wurde angemerkt, dass eine Generalisierung der erzeugten Kreaturen mit mehreren verschiedenen Körperausprägungen wünschenswert wäre. Während der Ausarbeitung dieses Forschungsprojektes, wurde jedoch aufgrund des Zeitmangels und der Limitationen der damit verbundenen Animationen und des Frameworks die Generierung der Kreaturen auf simple 2- und 4-Beiner beschränkt. Eine, wie eigentlich geplante Generalisierung der Erzeugung und des Trainings der Kreaturen würde weitreichende Vorteile mit sich bringen; zu Anfang müsste ein längeres Training auf großen Netzwerken ausgeführt werden und es somit ermöglichen ein Metalevel bzw. einen Standard zu setzen. Dieser Standard könnte einheitlich für eine große Menge an Kreaturen mit einer großen Variation an Körpermerkmalen genutzt werden, indem jede Kreatur, die innerhalb des durch das Training vorgegebenen Frameworks erzeugt werden würde, direkt durch dieses zur Laufzeit animiert werden könnte. Somit könnte also die Generation von Kreaturen mit vielfältigen Körpermerkmalen zur Laufzeit behandelt werden. Um diese jedoch erst gewährleisten zu können, müsste die in dieser Arbeit angeführte parametrische Erzeugung der Kreaturen ebenfalls weiter ausgebaut werden, um eine größere Palette von Körperausprägungen für Kreaturen zu ermöglichen.

Bei der Animation könnten weiterhin über die in dieser Arbeit vorgestellten Skills des Aufstehens und Laufens weitere Fähigkeiten von Kreaturen berücksichtigt und erlernt werden. Dabei ist jedoch nicht nur das Training, sondern darüber hinaus auch die Kombination der verschiedenen Skills von Kreaturen eine relevante Herausforderung, welche sich dann wiederum durch verschiedene Körperausprägungen (wie z.B. Flügel, Anzahl der Gliedmaßen, etc.) von Kreatur zu Kreatur unterscheiden könnten. Hier könnte vor allem in das Gebiet des Reward-Function-Engineering tiefer eingegangen werden.

Desweiteren könnte um über den Umfang der Projektgruppe hinauszugehen, graphische Aspekte deutlicher berücksichtigt werden. Dabei könnten beispielsweise Ray-Tracing zur Laufzeit oder eine Echtzeit-Graphik-Pipeline angewendet werden, um es auf ein heutiges Videospiel-Niveau zu heben. Somit würde ebenfalls das Thema des Texture-Mappings und des Parametrisierens von Texturen auf Körper betrachtet werden.



\chapter*{Themenallokation} \addcontentsline{toc}{chapter}{Themenallokation}

\begin{thallok}
	\item Einleitung: Thomas
	\item Grundlagen
	\begin{thallok}
		\item L-Systeme: Kay
		\item Metaball: 
		\item Reinforcement Learning: Jannik
		\item Cellular Automata: Markus
		\item Unity Features: 		
	\end{thallok}
	\item Verwandte Arbeiten: 
	\item Projektorganisation: Carsten, Thomas, Jannik, Leonard
	\begin{thallok}
		\item Creature Generator
		\item Creature Animator
	\end{thallok}
	\item Umsetzung
	\begin{thallok}
		\item Creature Generation: Markus
		\begin{thallok}
			\item Fachliche Umsetzung
				\begin{thallok}
					\item Parametrische Kreatur:
					\item L-System Creature Generation: Tom, Kay
					\item Mesh Generation: Leonard
					\item Automatisches Rigging: Leonard
					\item Skinning: Leonard
					\item Metaballs: Jona
					\item Jonas Creature Generation Method: Jona, Markus
				\end{thallok}
			\item Technische Umsetzung
				\begin{thallok}
					\item Unity-Package: Markus
					\item Konfiguration: Markus
					\item Bone-Definition: Markus
					\item Parametrische Generatoren: Markus
					\item Skeleton-Definition: Markus
					\item Skeleton-Assembler: Markus
					\item Mesh-Generator:
					\item Creature Generator: Markus
				\end{thallok}
		\end{thallok}
		\item Creature Animation: Nils, Carsten, Jan
		\begin{thallok}
			\item Fachliche Umsetzung
				\begin{thallok}
					\item Trainingsumgebung: Carsten
				\end{thallok}
			\item Technische Umsetzung
				\begin{thallok}
					\item Trainingsumgebung: Nils
					\item Erweiterung der Agent Klasse: Jan
					\item Konkrete Implementation des Agents:
					\item RL-Framework: Niklas
					\item LiDO3: Nils
				\end{thallok}
		\end{thallok}
		\item World Generation: Kay, Tom
		\begin{thallok}
			\item Dungeon Generierung: Tom
			\item Space Partitioning: Tom
			\item Terrain-Transformationen: Kay
			\item L-System Vegetation: Kay
		\end{thallok}
		\item Spiel: Markus
	\end{thallok}
	\item Evaluation
	\begin{thallok}
		\item Einschränkungen: Jannik, Nils, Niklas
		\item Ergebnisse: Nils, Jannik, Carsten, Niklas
		\item Diskussion: Niklas
		\item Probleme: Nils
	\end{thallok}
	\item Fazit \& Ausblick: Thomas
\end{thallok}


%
%	Hier werden Titel, Bearbeiter und das Datum eingetragen
%
%\newcommand\svthema{}
\newcommand\svperson{Jan Beier, Nils Dunker, Leonard Fricke, Niklas Haldorn, Kay Heider, \linebreak Jona Lukas Heinrichs, Mathieu Herkersdorf, Carsten Kellner, Markus Mügge, Thomas Rysch, Jannik Stadtler, Tom Voellmer}
\newcommand\svdatum{\today} % \today für das aktuelle Datum, sonst als Text eintragen
\newcommand\lvname{Projektgruppe: Entwicklung eines 3D RPG Videospiels mittels prozeduraler Inhaltsgenerieung und Deep Reinforcement Learning}
\newcommand\lvtyp{SS 2022 - WS 2022/2023}
\newcommand\lvinst{TU Dortmund}

%\hypersetup{ % Thema und Author in die Meta-Daten der PDF
%	pdftitle={Seminararbeit zum Thema: \svthema} 
%	pdfauthor={\svperson}
%}	


	
	% Hier wird der Titel-Bereich formatiert
	% Die zuvor definierten Textbausteine werden hier verwendet.
	\title{ \huge\textbf{Projektgruppe: \linebreak \linebreak Entwicklung eines 3D RPG Videospiels mittels prozeduraler Inhaltsgenerierung und Deep Reinforcement Learning} }
	\author{Jan Beier \and Nils Dunker \and Leonard Fricke \and Niklas Haldorn \and Kay Heider \and Jona Lukas Heinrichs \and Mathieu Herkersdorf \and Carsten Kellner \and Markus Mügge \and Thomas Rysch \and Jannik Stadtler \and Tom Voellmer}
	\date{\LARGE{\svdatum} \linebreak \linebreak \normalsize \centering \includegraphics[width=0.4\textwidth]{ressources/tu-dortmund.png} \linebreak \linebreak \large\textsc{"`\lvname"'} \linebreak \linebreak \large{\lvtyp} · \large{\lvinst}}
	
	\maketitle
	\thispagestyle{empty} % lässt die Seitennummer auf der Titelseite verschwinden
	\pagenumbering{Roman}